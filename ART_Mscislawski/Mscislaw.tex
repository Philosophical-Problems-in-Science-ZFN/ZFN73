\begin{artengenv}{Łukasz Mścisławski}
	{Is information ontological or physical, or is it perhaps something else? Some remarks on Krzanowski's approach to the concept of information}
	{Is information ontological or physical, or is it perhaps\ldots}
	{Is information ontological or physical, or is it perhaps something\\else? Some remarks on Krzanowski's approach to the concept of\\information}
	{Wrocław University of Science and Technology}
	{\label{mscislaw_start}As one may have noticed, the title of this paper is somewhat provocative. We found Roman Krzanowski's
	%\label{ref:RNDD4ZPbWpWxt}(2020a; 2020b; 2020c; 2022)
	\parencites*[][]{krzanowski_does_2020}[][]{krzanowski_what_2020}[][]{krzanowski_why_2020}[][]{krzanowski_ontological_2022} %
	 proposed approach to the problem of information very intriguing. Our aim here is to highlight some advantages when it comes to answering some fundamental questions in the philosophy of physics and metaphysics, as well as the philosophy of information and computer science. This issue is of great importance, so we propose that the introduction of some subtle distinctions between ontological and epistemological information can be regarded as being analogous to G.F.R. Ellis's analyses of the passage of time in his concept of the Crystallizing Block Universe 
	%\label{ref:RNDaS7kEBPadF}(Ellis and Goswami, 2012).
	\parencite[][]{ellis_space_2012}. %
	 This analogy could be useful when further studying the relations between different types of information. We also suggest some subjects for further study, ones where Krzanowski's proposal could serve as a~very solid foundation for examining traditional metaphysical issues by combining classical philosophical doctrines with the new approach.
	}
	{physical information, ontology, physics, philosophy of information.}


%\begin{document}
%\title{Is information ontological or physical, or is it perhaps something else? Some remarks on Krzanowski's approach to the concept of information}
%\maketitle
%
%Łukasz Mścisławski
%
%Politechnika Wrocławska
%
%\url{https://orcid.org/0000-0003-0195-906}
%
%\section*{Abstract}
%As one may have noticed, the title of this paper is somewhat provocative. We found Roman Krzanowski's
%%\label{ref:RNDD4ZPbWpWxt}(2020a; 2020b; 2020c; 2022)
%\parencites*[][]{krzanowski_does_2020}[][]{krzanowski_what_2020}[][]{krzanowski_why_2020}[][]{krzanowski_ontological_2022} %
% proposed approach to the problem of information very intriguing. Our aim here is to highlight some advantages when it comes to answering some fundamental questions in the philosophy of physics and metaphysics, as well as the philosophy of information and computer science. This issue is of great importance, so we propose that the introduction of some subtle distinctions between ontological and epistemological information can be regarded as being analogous to G.F.R. Ellis's analyses of the passage of time in his concept of the Crystallizing Block Universe 
%%\label{ref:RNDaS7kEBPadF}(Ellis and Goswami, 2012).
%\parencite[][]{ellis_space_2012}. %
% This analogy could be useful when further studying the relations between different types of information. We also suggest some subjects for further study, ones where Krzanowski's proposal could serve as a~very solid foundation for examining traditional metaphysical issues by combining classical philosophical doctrines with the new approach.

\section*{Introduction}
\lettrine[loversize=0.13,lines=2,lraise=-0.03,nindent=0em,findent=0.2pt]%
{T}{}his article primarily aims to concisely present Roman Krzanowski's approach to the concept of \textit{physical information.} This concept is perceived here as a~kind of ``intermediate concept'' that can act as a~bridge to the development of his concept of \textit{ontological information}
%\label{ref:RNDVGZ8jlBX2V}(cf. Krzanowski, 2020a; 2020b; 2020c; 2022),
\parencites[cf.][]{krzanowski_does_2020}[][]{krzanowski_what_2020}[][]{krzanowski_why_2020}[and especially][]{krzanowski_ontological_2022},%
and can be regarded as a~special case of the latter. Indeed, we believe that the latter concept inherits many characteristics from the former, but not all the issues have been solved.\footnote{It is also worth mentioning that in some works, Krzanowski seems to use these two terms (i.e., physical and ontological information) interchangeably
%\label{ref:RNDx2sgo6ehaO}(cf. Krzanowski, 2020b).
\parencite[cf.][]{krzanowski_what_2020}.%
} This situation implies that we are dealing with a~truly fundamental issue here, namely the fundamentality of the concept of information. Furthermore, this paper also aims to discuss some of this concept's interesting properties and present possible avenues for further developing this interesting proposal. It is well known that there is enormous interest in the philosophical aspects of the concept of information 
%\label{ref:RNDKUzkfsqSdA}(e.g. Adriaans and Benthem, 2008; Burgin, 2010; Floridi, 2011; Dodig Crnkovic and Burgin, 2019),
\parencites[e.g.,][]{adriaans_philosophy_2008}[][]{burgin_theory_2010}[][]{floridi_philosophy_2011}[][]{dodig_crnkovic_philosophy_2019}, %
 but we feel that Krzanowski's proposal is particularly attractive with some great research potential, especially for studying physical reality. It would %
therefore be helpful, in our opinion, to position this concept within the vast array of philosophical problems related to the notion of information, as well as introduce a~few distinctions to allow us to order the discourse.

The first distinction we would like to introduce, because we consider it important, is the distinction between two perspectives for the notion of information:
\begin{enumerate}[label=(\Alph*)]
\item Ontological; and
\item Epistemological.
\end{enumerate}
These two areas of research are by no means mutually exclusive, but they pose slightly different kinds of questions. Krzanowski
%\label{ref:RNDf3GlACHwY4}(2020c, pp.37–38; 2022, pp.123–151)
\parencites*[][pp.37–38]{krzanowski_why_2020}[][pp.123–151]{krzanowski_ontological_2022} %
 presented an analogous distinction.\footnote{A~slightly different version is also presented in 
%\label{ref:RNDU9el6H16Mu}(Krzanowski, 2020b),
\parencite[][]{krzanowski_what_2020}, %
 where a~distinction between abstract and concrete information is introduced. There is a~little more on this subject below.}
 For the purposes of this paper, however, they can be characterized as follows: The \textit{ontological} (A) perspective mainly poses questions about what something is, how it exists, what its inherent properties are, and so on. It is worth noting here that within this research area, it is possible to pose a~question about the general structure of reality (i.e., its ontology and the ``location'' of information within it) and thus its ontological status. Such a~general perspective makes it possible to regard information as something tangible, so there is no need to reify it, although we cannot exclude such a~possibility. This perspective is of particular interest to Krzanowski and therefore also to this paper.\footnotetext{As a~rule in this paper, ontology is not understood as a~network of concepts and relations between them, as is the case in some formal disciplines.}


The \textit{epistemological} (B) perspective, on the other hand, poses questions that are typical of epistemology, such as how something can be cognitively grasped and whether and how it relates to issues of knowledge, truth, cognition, and so on.

The second distinction that naturally arises when considering the notion of information relates to two approaches:
\begin{enumerate}[label=(\Roman*)]
\item Qualitative; and
\item Quantitative.
\end{enumerate}
It is plain to see that these two approaches can be combined with the above perspectives. We could say that the epistemological perspective (B) is compatible with both approaches (I and II). Similarly, it is clear that perspective A~must be compatible with approach I. Nevertheless, some interesting issues arise:

\begin{enumerate}
\item Can every characteristic of I~be expressed in the form of II? It seems that the answer to this question is by no means obvious, and trying to provide one may be better regarded as a~starting point for some interesting and deep research into addressing the general problem of relating the two approaches.
\item Is it possible to combine perspective A~with approach II for the concept of information? This is not something that could be achieved by simply declaring that ``information is quantitative only''. It seems that any serious attempt to answer this sort of question would possibly require connecting what we call \textit{information} with mathematical structures. Perhaps with regard to the ontological status of the latter, a~perspective along the lines of mathematical Platonism should be included here, at least to some extent.
\end{enumerate}
It is also worth noting another difficulty with approaches of type II, which can be formulated as follows: \textit{What is that} which is being quantitatively represented? This question becomes all the more pertinent when one accounts for Burgin's
%\label{ref:RNDXJtOCj8zkW}(2011, p.349)
\parencite*[][p.349]{burgin_information_2011} %
 observation, where the \textit{information} is something and the \textit{what} is a~measure imposed on it.\footnote{An analogy could be drawn with the subtle distinction between space and distance as an imposed measurement of it. It is otherwise surprising how many analogies from analyses of foundational physics are applicable to considerations of the concept of information.} Of course, it can always be argued that we are using a~projective definition here, but this would only serve to cut off the discussion instead of resolving the difficulty. In fact, this would be an ineffective ploy, because over 30 quantitative accounts of information can be given 
%\label{ref:RNDfjole41Xwa}(Burgin, 2010, pp.131–133).
\parencite[][pp.131–133]{burgin_theory_2010}. %
 Hence, two fundamental questions arise here: Firstly, which of these approaches should be adopted as a~starting point and why? Secondly, are these two approaches related to each other, and if so, how?

Thus, if we assume that quantitative approaches do not provide a~good basis for considering the concept of information, then it seems reasonable to attempt tackling this issue in a~different way. Quantitative approaches (II) seem to be strongly linked to the epistemological perspective of research (B), so we should perhaps place more emphasis on the qualitative approach (I) and try to look at it more from an ontological perspective (A). Roman Krzanowski's conception undoubtedly falls within such an area of research. It represents what, at least in a~sense, Floridi would call \textit{information for reality}
%\label{ref:RNDWSZkt20o32}(Floridi, 2011, pp.30–31).
\parencite[][pp.30–31]{floridi_philosophy_2011}.%
\footnote{We here use the label ``information for reality'' rather than Floridi's concept of information. In Floridi's work (cited above) there is an inconsistency between the definition and exactly using the notion of information in the case of ``information for reality''. We are very grateful to anonymous reviewer for bringing this to our attention.} The difference between Floridi's account and Krzanowski's proposal basically lies in how the latter approach concerns \textit{physical reality}. This observation requires further elaboration, because the problems concerning the relationships between physics and information are widely discussed. It should be noted, however, that these propositions are more quantitative in nature and approached from an epistemological perspective (B), which in a~way seems counter to Krzanowski's conception.

In the following discussion, the concept of physical information proposed by Krzanowski will therefore be presented, and its basic properties will be discussed. Potential avenues for future research will also be discussed.

\section*{The concept of physical information}


\begin{flushright}
\begin{footnotesize}
\vskip -1\baselineskip
``So Professor Isham, what is a~thing?''

``We can't say what is a~thing, but you can say what is not.''

``What is not?''

``Not what people think it is''

\vskip -.4\baselineskip
\parencite{medeiros_2005}
\end{footnotesize}
\end{flushright}
%(I, science, Winter 2005, p. 20)
%\vskip 2\baselineskip

The discussion quoted above may seem humorous, but this is a~common situation when fundamental concepts become the subject of research. In our opinion, research into the concept of information belongs to such research. Thus, the next step in facilitating a~deeper exploration of the notion of \textit{physical information} is to follow the example of Krzanowski in introducing a~distinction between \textit{abstract information} and \textit{concrete information}
%\label{ref:RNDFVJi9ciBl1}(cf. Krzanowski, 2020b, p.2).
\parencite[cf.][p.2]{krzanowski_what_2020}.%


The concept of \textit{abstract information} (IA) relates to some kind of cognitive activity, and based on Krzanowski's work, its main features can be expressed as follows
%\label{ref:RNDUbSu6SVSZt}(cf. Krzanowski, 2020b)
\parencite[cf.][]{krzanowski_what_2020}: %
\begin{enumerate}[label=(IA\arabic*)]
\item It is some cognitive agent's interpretation of physical stimuli, which may be a~signal, the state of physical system, or some other physical phenomenon.
\item It exists for a~cognitive agent, or it is at least relative to some agent, so it is agent-relative or ontologically subjective.
\item It has meaning for a~cognitive agent.
\item The notion of a~cognitive agent is understood here in a~very broad sense, such that it may be human, another biological system, or some artificially intelligent system.
\item The existence of IA indicates the presence of an abstract notion somewhere outside of space and time.
\end{enumerate}

When discussing the concept of information, the IA concept plays an important role. It seems that almost all the quantitative formulations we mentioned earlier can be assigned to this category of information, because they are in a~sense imposed on physical reality by the cognitive subject. Moreover, due to the research successes of physics, which employs mathematical methods to a~large extent, there is considerable temptation to narrow any discussion about the concept of information to references to physical reality.

Nevertheless, Krzanowski's concept of \textit{physical information} refers to the concept of information using the term \textit{concrete information} (IC). He refers to an extensive list of authors whose views converge in this respect
%\label{ref:RND0gu0V9TSmq}(e.g. Turek, 1978; von Weizsäcker, 1982; Nagel, 2012; Dodig-Crnkovic, 2013; Heller, 2014; Rovelli, 2016; Wilczek, 2016; Davies, 2020),
\parencites[e.g.,][]{turek_filozoficzne_1978}[von][]{von_weizsacker_einheit_1982}[][]{nagel_mind_2012}[][]{dodig-crnkovic_alan_2013}[][]{heller_elementy_2014}[][]{rovelli_meaning_2016}[][]{wilczek_beautiful_2016}[][]{davies_demon_2020}, %
 to name but a~few). He also seems to be guided by an opposition to attributing only the features of IA to information in general 
%\label{ref:RNDE2Cvw6yFhp}(cf. Krzanowski, 2020a; 2020c).
\parencites[cf.][]{krzanowski_does_2020}[][]{krzanowski_why_2020}. %
 This triggers a~need to introduce a~different approach to the concept of information, a~more qualitative one (IC). Thus, the fundamental features of IC can be described as follows 
%\label{ref:RND57FEnOikEz}(cf. Krzanowski, 2020b, p.2):
\parencite[cf.][p.2]{krzanowski_what_2020}:%
\begin{enumerate}[label=(IC\arabic*), start=0]
\item IC exists in space and time (i.e., spacetime) as a~physical object, which is why it is called \textit{concrete}.
\item With reference to IC0, IC is a~physical phenomenon, so it exists objectively and is not relative to anything.
\item IC has no intrinsic meaning.
\item IC is, in a~sense, responsible for the organization of the physical world.
\item IC's existence implies existence in the physical world, somewhere in the space-time continuum.
\end{enumerate}

The main goal behind introducing the IC concept is, according to Krzanowski, a~hope that it may unify multiple quantitative approaches, at least at a~conceptual level, or establish some order among the multiplicity of formulations. It therefore seems reasonable to conclude that his concept of \textit{physical information} refers to \textit{concrete information} that is ``\textit{associated with}'' the physical level of the organisation of matter. This statement requires some elaboration and clarification, however. It is also worth emphasizing that only with the concept of concrete information is there at least some way to use it within the general discourse about information, including possibly regarding information as meta-physical.

One can easily see how some may object to various properties of this concept, but since we engage in a~broader discussion of these properties later in this paper, it is more appropriate for now to continue presenting further key features of Krzanowski's proposal.

Undoubtedly, one of the most important features of \textit{concrete information} is its \textit{objective existence} (IC1). As Krzanowski puts it, this means it exists as a~physical phenomenon or object, independently of any observing agent
%\label{ref:RNDJUFLLNhpxC}(cf. Krzanowski, 2020b, pp.4–5).
\parencite[cf.][pp.4–5]{krzanowski_what_2020}.%


The second feature emphasized by Krzanowski is IC's lack of intrinsic meaning (IC2), referring to how meaning is derived from an observed reality (e.g. a~physical object, phenomenon, etc.) by a~cognitive agent. Since we are here discussing physical reality in itself, this means it has no meaning of its own. This reality can be interpreted from many points of view, but the procedure of deriving meaning is actually a~shift into the realm of \textit{abstract information}
%\label{ref:RND5ylgVXCfWX}(cf. Krzanowski, 2020b, pp.5–6).
\parencite[cf.][pp.5–6]{krzanowski_what_2020}.%


The presentation of the third feature is very, possibly even hopelessly, difficult. As Krzanowski emphasizes, any discussion of the concept of information becomes interwoven with notions such as \textit{form}, \textit{structure}, \textit{object}, and so on. Another fundamental problem here emerges when one tries to understand what it means for \textit{information} to be \textit{associated} with concepts like form and structure. To some extent, however, we can say that IC is in some way responsible for the organization of matter. This statement needs clarification, which Krzanowski provides by addressing the question of whether IC can be considered a~physical phenomenon
%\label{ref:RNDPU4TwRqmMg}(cf. Krzanowski, 2020b, pp.3–4).
\parencite[cf.][pp.3–4]{krzanowski_what_2020}. %
 More specifically, he describes \textit{physical information} binding with physical reality and mathematics (or mathematical structures) as follows:
 \newlist{mscislaw}{enumerate}{2}
 \setlist[mscislaw,1]{label=(PhI\arabic*)}
 \setlist[mscislaw,2]{label=(PhI\arabic{mscislawi}.\alph*)}
\begin{mscislaw}
\item  Physical information, as an inheritor of concrete information, is described as a~physical phenomenon. It should be highlighted that Krzanowski gives a~very special meaning to this statement, because physical information being a~physical phenomenon implies that this special type of information is an irreducible aspect of physical reality. In a~way, it recognizes something—whether it be the form, structure, or organization of some entity—as a~purely physical phenomenon in itself.

\item  Physical information exhibits properties that can be attributed to physical entities, namely that it:
\begin{mscislaw}
\item is observable;
\item is ontologically objective;
\item can be manipulated;
\item has no intrinsic meaning; and
\item can be quantified or measured.
\end{mscislaw}
\item  Physical information is not a~mathematical or physical structure, thus preventing it from being considered as part of the realm of mathematical or physical structures, something that could easily lead to referring to such structures rather than to the (physical) information itself. However, where physical reality exists, there must also exist physical information
%\label{ref:RNDHah1cfOcK0}(cf. Krzanowski, 2020b, pp.3–4, 6).
\parencite[cf.][pp.3–4]{krzanowski_what_2020}.%
\end{mscislaw}

What is also significant is how Krzanowski does not insist that his concept is well-defined without any ambiguities. In contrast, he emphasizes that in any serious analysis of the concept of information and its relation to the physical world, there will be ambiguities. Moreover, such ambiguity is characteristic of how physical reality and information are related. Like Krzanowski, we believe that remaining at a~more or less descriptive level is unavoidable when addressing such a~subtle and intangible issue, thus excluding any narrow perspective that someone could subjectively call ``sensible''
%\label{ref:RNDgplFmjoqUd}(cf. Krzanowski, 2020b, p.6).
\parencite[cf.][p.6]{krzanowski_what_2020}.%


\section*{Remarks and potential avenues for further development}
Starting with some additional remarks, we refer to the epistemological perspective for information and quantitative approaches. We begin with the obvious statement that any cognitive act is possible if and only if a~cognitive agent can cognize something. This leads to the following statement: In the reality in which we are able to cognize, there exist entities such as cognitive agents and entities that they can cognize.\footnote{We do not want to start a~battle here about whether they are able to build knowledge about their reality but rather state that they cognize without discussing what knowledge is. We simply need to assume that there are various stimuli that agents can perceive and react to in a~certain way.} If their existence is long and stable enough, then an act of cognizance is possible. We believe that this situation strongly suggests that some structures must exist in physical reality, that there are cognitive agents at a~physical level, and that these agents can perceive the structures in some way. This then leads us to the conclusion that the structures of physical reality precede acts of cognition. Thus, it seems that all quantitative approaches and epistemological perspectives for research into the concept of information come secondary to one very fundamental fact: There are physical structures. We understand physical structure in a~very broad manner as something that can be in some way distinguished from its background. For example, in this sense, even an elementary particle can be regarded as a~structure, because it can be viewed as an excitation (or an excited state) of the quantum field. We are not suggesting here that it should be understood as something that is separated from its background but rather that we are allowed to say that \textit{there is a~physical structure if there is any differentiation in a~considered physical reality}, regardless of whether we can describe this differentiation mathematically or not.\footnote{There are two interesting properties of such an approach: 1) We are completely free in referring to physical objects as parts of wider structures, and 2) we are free to regard physical objects as entities of internal structure, even though it may be infinitely complex.} In our opinion, to answer questions about how this is possible, one of the most obvious ways would be to point out how the laws of physics tell matter how to behave. However, we here encounter extremely difficult questions about the relations between matter, information, and the laws of physics. Nevertheless, we would like to emphasize that much depends on how we understand the laws of physics. The first option would be to define the laws of physics as part of our description of the regularities in physical reality (PLE).\footnote{We treat physical laws here as scientific laws.} This means that human beings observing these regularities of physical reality act as cognitive agents trying to express these regularities using mathematical structures. However, this returns us to the epistemological perspective. The second option lies in the definition of the laws of physics (PLO), such that we could assume that a~kind of Platonic realm for mathematical structures exists, and a~portion of those structures govern and shape matter
%\label{ref:RNDp7aw9XSjdj}(e.g. Heller, 1998; Penrose, 2006; see also Grygiel, 2022).
\parencites[e.g.,][]{heller_czy_1998}[][]{penrose_road_2006}[see also][]{grygiel_critical_2022}. %
 Such a~possibility immediately opens a~door to the ontological perspective,\footnote{We have to admit here that opening such a~door also opens up a~Pandora's box of questions about mutual relations, such as the ``Platonic'' world of mathematical structures, matter, the mind, and so on, but we will skip over this endless discussion here.} and within this context, we would like to emphasize the importance of the ontological perspective in research into the concept of information 
%\label{ref:RNDI3n3XHXXN0}(cf. Krzanowski, 2020c, p.53).
\parencite[cf.][p.53]{krzanowski_why_2020}. %
 Nevertheless, we are interested in attempting to answer the question of why regularities in physical reality can exist at all? This question involves the relationships between matter, information, and the laws of physics, and it is all the more difficult because all those concepts are highly problematic. This is exactly why we regard Roman Krzanowski's concept of physical information as such an interesting proposal. It seems to make it possible to at least partly answer Hawking's famous question: \textit{What is it that breathes fire into the equations and makes a~universe for them to describe}? 
%\label{ref:RNDUB8505EguL}(Hawking, 1988, p.174).
\parencite[][p.174]{hawking_brief_1988}.%


\paragraph{R1:} Our first remark refers to the feature PhI3 and the possible relations that \textit{physical information} has with physical and mathematical structures. Krzanowski claims that where physical reality exists, there is physical information, and this suggests two possibilities:
\begin{enumerate}[label=(\alph*)]
\item Physical information is something inherent in matter, but this solution excludes any further discussion of the laws of physics, the possibility of cognizing physical reality, and so on.\footnote{We dare to posit that such a~solution is unsatisfying and of little interest. However, it still leaves us with unanswered questions: What is matter? Why is it formed in the way it is?}
\item Physical information is s\textit{omehow different from physical reality}, which is the research domain of physicists. Nevertheless, it is \textit{somehow} \textit{associated with} \textit{it}, albeit with a~different ontological status. It reveals itself through the existence of physical structures and the opportunity to cognize physical reality, even with measurement. Thus, we regard this as a~strong case for regarding it as a~\textit{meta-physical} reality, one that is tangible because it is ``responsible for'' creating physical structures. This last claim in some way justifies thinking of physical information as something \textit{inherent} to any physical object or phenomenon as an ``internal'' (meta-physical) constituent of it. Additionally, belonging to the ontological level and existing prior to any cognitive agent, physical information turns out to be more fundamental than any quantitative definition of information, albeit with the caveat contained in R2.
\end{enumerate}

\paragraph{R2:} It seems to us that PhI3 suggests that, in some way, the concept of physical information is not to be regarded as a~concrete mathematical structure. We would like to point out that this feature of the discussed concept needs further development. We claim that at the moment, the concept of physical information and its relations strongly depend upon what ontological assumptions are made, such as what is assumed to exist, whether there is any kind of metaphysical pluralism, and how particular types of entities (or structures) interrelate. For example, if a~kind of Platonic ontological structure of reality is assumed,\footnote{A~good example of such a~structure of reality was described by Penrose
%\label{ref:RNDHobtHTUq2h}(2006, pp.18–19).
\parencite*[][pp.18–19]{penrose_road_2006}.%
} it could also be assumed that a~part of the objectively existing Platonic mathematical world completely models (or causally acts upon and determines) physical reality, so physical structures are merely a~material representation of particular mathematical structures. In such a~case, physical information would surely be associated with certain mathematical structures, and this could be considered within any quantitative approach. There are of course easy way to escape this difficulty. More specifically, it suffices to assume that the mathematical structures of a~Platonic world do not describe the entirety of physical reality.\footnote{Such an example was also described by Penrose 
%\label{ref:RND0ROIWr74V2}(2006, pp.19–21).
\parencite*[][pp.19–21]{penrose_road_2006}.%
}

\paragraph{R3:} Because physical information in some way inherits the features of concrete information, there is some difficulty with IC0, IC4, and PhI1. All these features suggest that physical information is ``located'' on the Newtonian-like stage of space and time. However, it seems that as Krzanowski describes it, space and time (or spacetime) are independent of physical information. General Relativity, however, describes spacetime (or space and time) as part of physical reality. Hence, this strongly suggests that we should regard physical information as something that is also in some way associated with the structure of spacetime. In other words, it ``contains information'' for spacetime. This aspect of the concept proposed by Krzanowski gives further backing for regarding physical information as something meta-physical while still being strictly connected to, or associated with, the physical level of the organization of matter.

\paragraph{R4:} In all the key works
%\label{ref:RNDQIgL3jHQ2h}(i.e. Krzanowski, 2020b; 2020c; 2022),
\parencites[i.e.,][]{krzanowski_what_2020}[][]{krzanowski_why_2020}[][]{krzanowski_ontological_2022}, %
 there is a~lack of any linkage to quantum theories (e.g. quantum mechanics, quantum field theory, etc.). There is also no remark about no-go theorems (e.g. Kochen-Specker theorem), contextuality, and so on. These omissions are a~little bit puzzling, but they could be explained in two ways:
\begin{enumerate}[label=(\arabic*)]
\item The concept of physical information refers to the very physical reality (cf. IC3 and PhI1), and as such, it also refers to the strange quantum realm. There is therefore no need to confer a~special status to this realm or any issues connected with the strange quantum features of it.

\item As with all physical theories, quantum theories are not ultimate theories but rather something through which we try to describe and explain physical reality. In this respect, it is a~purely epistemological perspective, while the subject of interest (the concept of physical information) adopts an ontological perspective. However, in this case, a~question arises as to whether any suggestions from philosophical research into quantum theories should be considered when exploring the concept of information, particularly for physical information. This matter requires extreme caution, however, because scientific theories tend to evolve relatively quickly, so drawing any far-reaching ontological conclusions from them is a~difficult undertaking. We therefore regard this issue as an open question.
\end{enumerate}
\paragraph{R5:} By virtue of IC3 and PhI1, the concept of physical information refers to physical reality, but it is not clear whether it is associated with the entirety of physical reality or just particular structures. In the latter case, a~question naturally arises about the relations in which particular ``physical information'' is associated with particular structures. On the other hand, it is precisely this reference to physical reality that allows the concept under discussion to be open to being ``contained'' in concepts of information, meaning higher levels of organisation of matter, such as chemical, biological, and so on. This opens up a~very interesting research area that relates to the possible types of information, their mutual relationships, and particularly the complexity (highly complex, non-linear, and chaotic systems) of it all. Indeed, Krzanowski
%\label{ref:RND0P5ufCGsH6}(2020b, p.13)
\parencite*[][p.13]{krzanowski_what_2020} %
 recognizes these areas. Another interesting question refers to potential relations with issues connected with computer science and natural computation. It again seems by virtue of the fact that physical information refers to physical reality, it could be included in such analyses. This problem is partially addressed by Krzanowski 
%\label{ref:RNDEEWlvPjTy4}(2022),
\parencite*[][]{krzanowski_ontological_2022}, %
 but we believe that it warrants further research, especially within the context of computations and relations between physical structures, some of which are very special, such as computing devices\footnote{By computing devices, we refer to artificial computing devices like personal computers, while we assume that any natural process can be regarded as a~form of computation, so any ``natural computing'' device is a~natural process.} and mathematical structures.

\paragraph{R6:} If physical information is to be regarded as something that refers to structures in nature
%\label{ref:RND0usaVffGie}(Krzanowski, 2022, pp.86, 91–92),
\parencite[][pp.86]{krzanowski_ontological_2022}, %
 we should also account for the following issues:
\begin{enumerate}[label=(\alph*)]
\item Physical structures are dynamic, so we should try to answer the following question: Are changes in physical structures really also changing the physical information, so it should be regarded as dynamic. Or does physical information contain the ``dynamics'' of these physical structures? Is it perhaps rather the case that changes in physical reality, caused naturally or otherwise, take place in a~manner that is determined by physical information?
\item Perhaps it is a~good idea to regard physical information as something ``standing behind'' physical structures and their dynamics (changes). One possibility would be to view physical information as a~kind of potentiality for creating (physical) structures.\footnote{From private correspondence with R. Krzanowski.} This could be based on the fairly obvious observation that structures exist in nature, and nature tends to create structures, yet the idea needs further research, because the potential to create structures does not necessarily stem from the fact that there has to be structures.\footnote{It should be noted that this issue was addressed by Czesław Białobrzeski, among others. To explain how it is possible for structures to arise in nature, Białobrzeski adapted the ontological ideas of Nicolai Hartmann and introduced the category of organisation (Polish: \textit{kategoria ustrojowości}), which is responsible for allowing higher layers of reality to arise, as well as a~real factor that he called potentiality that is responsible for the state and organisation of a~system
%\label{ref:RNDzTObVw8edO}(cf. Białobrzeski, 1984, pp.243–247; Mścisławski, 2017).
\parencites[cf.][pp.243–247]{bialobrzeski_podstawy_1984}[][]{mscislawski_miedzy_2017}. %
 This area of research is closely connected to the issue of the relation between physical information and complexity (see R5 above).}

\item It seems that physical information, by virtue of being ``responsible'' for manifesting physical structures, makes it possible for epistemic concepts of information to exist.
\end{enumerate}
\paragraph{R7:} A~subtle issue arises when we question the ontological status of physical information, as well as its genesis. Indeed, there are many opportunities for further research in this area
%\label{ref:RNDHDALPOZtBU}(cf. Krzanowski, 2020b, p.13).
\parencite[cf.][p.13]{krzanowski_what_2020}. %
 There is also a~very important question about the causal relations between physical information and the physical reality with which it is connected. Within this context, the problematic relations between matter, physical information, and the laws of physics arise once more 
%\label{ref:RNDtQjmz6zqlx}(e.g. Davies, 2007).
\parencite[e.g.,][]{davies_implications_2007}.%


As has been presented thus far, physical information is described as something that ``stands behind'' physical structures, while many points suggest that it has no physical character of its own
%\label{ref:RND4yIr0Pvgg0}(see also Burgin, 2017).
\parencite[see also][]{burgin_general_2017}. %
 It therefore seems quite natural to treat physical information as being metaphysical or, in other words, ontological information 
%\label{ref:RND4GJ728EdF2}(cf. Krzanowski, 2020b; 2022).
\parencites[cf.][]{krzanowski_what_2020}[][]{krzanowski_ontological_2022}. %
 Such a~move makes it possible to view this information as referring to physical systems rather than being a~physical phenomenon in itself. As Krzanowski puts it, ontological information is not something from the Platonic world but rather something that is closely connected with physical reality, with it unveiling itself much like physical phenomena and their properties do 
%\label{ref:RNDmqCuTs4leU}(cf. Krzanowski, 2022, p.110).
\parencite[cf.][p.110]{krzanowski_ontological_2022}. %
 Yet another question arises here, however: How does ontological information relate to the laws of physics? If we assume that we define these laws as PLE, the solution is relatively simple. The real problem arises when we define these laws as PLO, and this represents another potential area for further research. What is also interesting here is that this step also positions information as possibly having two modes of existence, namely concrete and abstract 
%\label{ref:RNDYZbAiVM2On}(Krzanowski, 2022, pp.154, 223),
\parencite[][pp.154]{krzanowski_ontological_2022}, %
 so it does not ultimately solve the fundamental difficulties of their relations to spacetime, the problem of causality relations, and the issue of complexity.

We would like to suggest another potential research area that addresses the issue of treating physical information (and ontological information) as being associated with a~very special kind of transition, namely the transition from ontological possibility to a~concrete physical structure (reality).\footnote{This proposal is analogous to Ellis' proposal of understanding ``now'' as the transition from the future, which is understood as ontological undetermination (uncertainty), to the past, which is understood as epistemological uncertainty
%\label{ref:RNDhcFg6liBRN}(cf. Ellis and Goswami, 2012).
\parencite[cf.][]{ellis_space_2012}. %
 In this approach to the concept of physical information, it is also important that we refer to a~transition from ontological possibilities to actual emerging physical states of reality, and we are not referring just to information about possibilities (initial possible states) and information about actuality (final actual states). We see a~certain similarity between this transition and the proposed mechanism of decoherence for solving the problem of vector state reduction in quantum mechanics 
%\label{ref:RNDqADKlmWgNg}(cf. Zurek, 2002).
\parencite[cf.][]{duplantier_decoherence_2002}.%
} Is physical information therefore to be regarded as a~transition, a~kind of ``ontological process'' that is analogous to the forming of matter in hylomorphism or rather as an analogy of form 
%\label{ref:RNDlUKgDa35RO}(cf. Krzanowski, 2020b, p.13)?
\parencite[cf.][p.13]{krzanowski_what_2020}? %
 Or should it be regarded as a~kind of description or algorithm for another factor acting on matter?\footnote{If this is the case, we would rather regard physical information both as a~description (data) and a~kind of algorithm. It would determine the features of structures, their behaviour (dynamics), and both a~physical and ontological manifestation.} Perhaps further research into the abovementioned transition could shed some light on the links between physical or ontological information and causality. Indeed, these open questions could be a~starting point for further research.

In our opinion, all the remarks mentioned above lead us to yet another potential area of research, and the question addressed within it could be formulated as follows: What kind of ontology would be extensive enough to encompass all possible types of beings\footnote{As a~kind of being (something that exists), we also refer here to structures of any type and kind.} and existence, such that it could deal with all the complex issues? The situation becomes even more complicated if we also include the issue of virtual beings
%\label{ref:RNDSZtUvcTGF2}(e.g. Skowron, 2020)
\parencite[e.g.,][]{skowron_virtual_2020} %
 and relations between virtual reality (or realities) and physical reality. It seems that a~kind of combined ontology may be needed, such as one based on the proposal of Perzanowski 
%\label{ref:RNDpp60i4SLIj}(2015).
\parencite*[][]{perzanowski_jest_2016}.%


\section*{Conclusion}
We find Krzanowski's proposed concept of physical information very interesting, particularly at a~certain stage in his study of the concept of information. While there are many points in which this concept seems to be ambiguous, there are also some interesting areas for possible further research. Thus, we have endeavoured to present the concept and point out the problematic aspects. Most of these could be regarded as potential starting points for further study, despite the fact that some of these ambiguities were partially clarified in the concept of ontological information presented by Krzanowski
%\label{ref:RNDgvjMJHNDo8}(2022).
\parencite*[][]{krzanowski_ontological_2022}. %
 Nevertheless, presenting this concept in light of the issues presented above warrants a~separate study.

\paragraph{Acknowledgements.}
We would like to express our gratitude to the two anonymous reviewers for their careful reading of our manuscript and extremely valuable and helpful comments.

\end{artengenv}
\label{mscislaw_stop}
