
\begin{editorialeng}{Roman Krzanowski}
	{Editorial note}
	{Editorial note}
	{Editorial note}
	%	{Copernicus Center for Interdisciplinary Studies}
	{Philosophy of economics -- a~school of pluralism and humility}


%\section{Editorial Note}
\lettrine[loversize=0.13,lines=2,lraise=-0.03,nindent=0em,findent=0.2pt]%
{T}{}his special edition of Philosophical Problems in Science (\textit{Zagadnienia Filozoficzne w~Nauce} or ZFN) focuses on concepts of information and computing. On reading this issue, you may be surprised by the absence of traditional perspectives and themes that one would usually expect from such collections, but this apparent oversight is deliberate. The eight papers collected in this special edition of ZFN bring together perspectives that aim to inspire readers rather than confirm concepts that have already been researched. The main motivation behind this collection is a~desire to explore the philosophical dimensions of computing and information sciences. Thus, for anyone looking for new ideas related to the philosophy of computing and information and wondering what is on the horizon, this special edition of ZFN may be the place to start.

The collection begins with a~paper by Gordana Dodig-Crnkovic (Chalmers University of Technology) entitled ``In Search of a~Common, Information-processing, Agency-based Framework for Anthropogenic, Biogenic, and Abiotic Cognition and Intelligence.'' This paper aims to provide a~general introduction to advances in natural computing and information processing in order to:
\myquote{
better to understand mechanisms of cognition and intelligence as they appear in nature. New understandings of information and processes of physical (morphological) computation contribute to novel possibilities that can be used to inspire the development of abiotic cognitive systems (cognitive robotics), cognitive computing and artificial intelligence.
}
This paper also includes extensive, up-to-date references that will help those wishing to explore this topic further by serving as a~guide to the state of current research in natural computing.

Next comes a~paper by Javier Toscano (Technische Universität Chemnitz) entitled ``But Seriously: What Do Algorithms Want? Implying Collective Internationalities in Algorithmic Relays---a Distributed Cognition Approach.'' This paper presents the concept of algorithms not as it is usually conceived, namely as a~sequence of logical steps, but rather as a:
\myquote{
larger construct that draws upon sociological and anthropological theories that underline social practices to propose expanding our understanding of an algorithm through the notion of ``collective internationalities.''
}
This paper contributes to the discussion about the role that ``intentionalities play in understanding socio-structured practices and cognitive ecologies.'' Furthermore, an extensive bibliography offers up-to-date sources on the paper's topic.

Another viewpoint for the fundamental concepts of computing is put forward by Alice Martin, Mathieu Magnaudet, and Stéphane Conversy (Interactive Informatics Team of ENAC Research Lab) in a~paper entitled ``Modelling Interactive Computing Systems: Do We Have a~Good Theory of What Computers Are?'' This paper discusses the conceptualization of interactive computer systems. According to the authors, this type of computing does not receive enough attention from philosophers and computer scientists, so their paper attempts to fill this gap. The paper surveys three areas in which interaction models can be framed: works on concurrency by Milner, works on reactive Turing machines, and works on interaction as a~new computing paradigm. For each of these models, the authors present the motivation behind it, summarize its accounting of interaction and its legacy, and point out issues related to our understanding of computers. The provided references also provide a~detailed review of the available literature for this topic.

The interdisciplinary approach to the philosophy of information is a~key topic of the next paper by Hyungrae Noh (Sunchon National University), which is titled ``Shannon-Inspired Information in the Clinical Use of Neural Signals Concerning Post-Comatose Patients.'' This paper links the two domains of medicine and the philosophy of information. The author posits that the current clinical methods for identifying a~minimally conscious state in patients based on behavioral assessments may not recognize signs of executive function in post-comatose patients. The author suggests that clinicians should instead look to localized brain ``activities in response to task instructions, such as imagining wiggling toes, to diagnose minimal consciousness.'' The author further suggests that the proposed method is more objective and reliable, because it does not require language comprehension, which may be severely impaired for patients in a~minimally conscious state. This paper opens up new perspectives on the philosophy of information as applied philosophy, and as with all good papers, the references provide a~detailed review of the related literature.

The discussion around the fundamental issues of the philosophy of information is the topic of a~paper by Łukasz Mścisławski (Wrocław University of Science and Technology) entitled ``Is Information Ontological or Physical, or Is It Perhaps Something Else? Some Remarks on Krzanowski's Approach to the Concept of Information.'' The paper presents a~critical evaluation of the concept of physical information that has been proposed by Roman Krzanowski. According to Mścisławski, the concept of physical information may play an important role in the philosophy of physics and metaphysics, the philosophy of information, and computer science. The author further states that the distinctions between ontological, which is another term used to denote physical information, and epistemological ``information can be regarded as being analogous to G.F.R. Ellis's analyses of the passage of time in his concept of the Crystallizing Block Universe.'' For anyone wanting to become familiar with the concept of physical information and its potential implications for cosmology, physics, and computing, this paper is a~good place to start.

The next paper in the collection was penned by Kristina Šekrst and Sandro Skansi (University of Zagreb), and it is entitled ``Machine learning and essentialism.'' This paper studies the connection between machine learning and essentialism. The authors posit that similarity-based approaches are more suited for pattern recognition and ``complex deep-learning issues, while for classification problems, mostly for unsupervised learning, essentialism seems like the best choice.'' The authors conclude that essences are not present in data but rather in learned targets, so machine learning does not provide any evidence for the independent existence of essential properties. Thus, our experiences with machine learning, according to the authors, do not offer any proof to support the ontological status of essences. A~substantial list of references related to essentialism and machine learning is provided at the end of the paper.

A~complementary view about the ontological commitments of artificial intelligence is presented in the paper written by Roman Krzanowski and Paweł Polak (Pontifical University of John Paul II in Kraków) entitled ``The Meta-Ontology of AI systems with Human-Level Intelligence.'' Meta-ontology in philosophy is a~discourse centered on ontology, ontological commitment, and the truth condition of ontological theories. The authors posit that the meta-ontology of current AI systems is concerned with computational representations of reality in the form of structures, data constructs, and computational concepts, while the ontological commitment of AI systems with human-level intelligence must be directed at what exists in the outside world. This paper builds upon the ontological postulates that were formulated by Brian Cantwell Smith about AI systems, and an extensive list of relevant literature is also of course provided.

The final paper was written by Krzysztof Sołoducha (Military University of Technology), and it is titled ``Analysis of the Implications of the ‘Moral Machine' Project as an Implementation of the Concept of ‘Coherent, Extrapolated Volition' for Building Clustered Trust in Autonomous Machines.'' This paper focuses on performing an ``analysis of Eliezer Yudkowsky's concept of ‘coherent extrapolated volition' (CEV) as a~response to the need for a~post-conventional, persuasive morality that meets the criteria of active trust in the sense of Anthony Giddens.'' In the paper, the ``authors formulate guidelines for transformation of the idea of a~coherent extrapolated volition into the concept of a~coherent, extrapolated and clustered volition.''In the author's words, ``The argumentation used in the paper is intended to show that the idea of CEV transformed into its clustered version can be used to build a~technically and socially efficient decision-making pattern database for autonomous machines.'' As with any excellent paper, an extensive list of relevant resources is provided.

In addition to the eight abovementioned papers, there are two essays. These differ from the papers by presenting a~more open perspective that allows for some personal views that would likely be too tentative for a~formal work. The essay format therefore allows authors to share creative ideas beyond formal hypotheses and present the reader with some inspiring and challenging reading.

The first essay by Kazimierz Trzęsicki is titled ``Perspective on Turing Paradigm.'' It argues that Turing planted the seeds of a~new paradigm in which the \textit{book of nature} is written in algorithms. In his arguments, the author delves far into the past, touching upon the works of the Babylonians and Egyptians, as well as later figures like Roger Bacon, Nicolas de Condorcet, Galileo, Leibnitz, and many others. The value of this paper lies in how the author tries to connect all of these past, geographically dispersed thinkers with modern ideas. Nevertheless, the success of this approach should be judged by the reader. The concepts and personalities collected in this essay is so extensive that Turing himself would have been surprised by how many people contributed to his ideas. After all, it is hard to be original!

The second essay was written by Adam Olszewski (Pontifical University of John Paul II in Kraków), and it is titled ``Will a~Human Always Outsmart a~Computer? An Essay.'' The author presents the model for the ``outsmarting'' of a~machine by a~human based on a~mathematical game between two players (the base domain), such that winning the game is denoted as ``outsmarting.'' The game in question is similar to a~Banach-Mazur game. The author concludes that while in the gaming example, a~man beats the hypothetical machine, the question is then this: How far can the results of this thought experiment be generalized? A~rather frugal reference list gives sufficient links to sources for those less familiar with the discussed ideas.

Finally, we have three book reviews: The first review by Paweł Polak concerns Roman Krzanowski's (2021) book \textit{Ontological information: information in the physical world} (Hackensack, New Jersey: World Scientific). This review is a~sort of addendum to Mścisławski's previously mentioned paper, and it exposes the philosophical underpinnings of Krzanowski's book and the perspectives it opens up. The second review, also by Paweł Polak, is for Andrzej Bielecki's (2019) book \textit{Models of Neurons and Perceptrons: Selected Problems and Challenges} (Cham: Springer International Publishing). Bielecki's work makes important contributions to contemporary philosophy in science by showing the role of computing in mathematizing subcellular biology. The third book review by Łukasz Mścisławski concerns a~book by Cappelen and Dever (2021) entitled \textit{Making AI Intelligible. Philosophical Foundations} (Publishing: Oxford University Press). This book examines possible ways to make AI intelligible, and many questions remain to be asked about this from a~philosophical perspective.

With thirteen works in the form of papers, essays, and book reviews, this special edition of ZFN represents a~fairly substantial package of ideas and concepts. No one is obliged or expected to read all these works, but whatever essays or papers the reader chooses to digest will likely be greatly rewarding.

Last but not least, we would like to acknowledge the excellent work of this ZFN edition's editors, Paweł Polak and Piotr Urbańczyk. Without their dedication and long nights of effort, this publication would not have been possible.

Some of the papers collected in this edition of ZFN were presented at the Philosophy in Informatics VI: Frontiers of Philosophy of Computing and Information conference held on December 16–17 and organized by the Polish Academy of Arts and Sciences (PAU) 2021. We would like to thank Gordana Dodig-Crnkovic for supporting this conference and contributing to this special edition.

\begin{flushright}
Roman Krzanowski\\
Editor of this special ZFN collection
\end{flushright}

\end{editorialeng}
