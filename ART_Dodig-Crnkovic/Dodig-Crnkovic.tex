\begin{artengenv}{Gordana Dodig-Crnkovic}
	{In search of a~common, information-processing, agency-based framework for anthropogenic, biogenic, and abiotic cognition and intelligence}
	{In search of a~common, information-processing, agency-based\ldots}
	{In search of a~common, information-processing, agency-based\\framework for anthropogenic, biogenic, and abiotic cognition\\and intelligence}
	{Chalmers University of Technology,\\University of Gothenburg\\and Mälardalen University, Sweden}
	{Learning from contemporary natural, formal, and social sciences, especially from current biology, as well as from humanities, particularly contemporary philosophy of nature, requires updates of our old definitions of cognition and intelligence. The result of current insights into basal cognition of single cells and evolution of multicellular cognitive systems within the framework of extended evolutionary synthesis (EES) helps us better to understand mechanisms of cognition and intelligence as they appear in nature. New understanding of information and processes of physical (morphological) computation contribute to novel possibilities that can be used to inspire the development of abiotic cognitive systems (cognitive robotics), cognitive computing and artificial intelligence.
	}
	{information, computation, cognition, intelligence, extended evolutionary synthesis, anthropogenic, biogenic and abiotic cognition.}


\section*{Information, computation, cognition, intelligence, and evolution of living organisms}
The notion of information is used nowadays not only to refer to means of communication between humans, but also to denote data structures utilized for communication by other living organisms, even the simplest ones like single cells as used in the fields of bioinformatics or neuroinformatics.

In what follows we build on the ideas presented in
%\label{ref:RND9pdQ8PORXA}(Dodig-Crnkovic, 2017a):
\parencite[][]{dodig-crnkovic_computational_2017}:%


\myquote{
a~view of nature as a~network of info-computational agents organized in a~dynamical hierarchy of levels. It provides a~framework for unification of currently disparate understandings of natural, formal, technical, behavioral, and social phenomena based on information as a~structure, differences in one system that cause the differences in another system, and computation as its dynamics, i.e., physical process of morphological change in the informational structure.

}
In the current definition of computation as a~dynamic of information, computation is taken to be any process of information transformation that leads to behavior, and not only those processes that we currently use to calculate, manually or with a~machinery:

\myquote{
Traditionally, the dynamics of computing systems, their unfolding behavior in space and time has been a~mere means to the end of computing the function which specifies the algorithmic problem which the system is solving. In much of contemporary computing, the situation is reversed: the purpose of the computing system is to exhibit certain behaviour. [...] We need a~theory of the dynamics of informatic processes, of interaction, and information flow, as a~basis for answering such fundamental questions as: What is computed? What is a~process? What are the analogues to Turing completeness and universality when we are concerned with processes and their behaviors, rather than the functions which they compute?
%\label{ref:RNDDGoXzTDsU7}(Abramsky, 2008)
\parencite[][]{abramsky_information_2008}%


}
Cognition can be defined as a~process of ``being in the world'' of an agent. For living organisms, cognition is a~process of life (perception, internal process control by information, actuation/agency),
%\label{ref:RNDoh9e7EeRCW}(Maturana, 1970; Maturana and Varela, 1980; Stewart, 1996).
\parencites{maturana_biology_1970}{maturana_autopoiesis_1980}{stewart_cognition_1996}.
Cognition of an organism is based on the ability to learn from the environment and adapt so as to survive as an individual and as a~species, for which organisms use information and its processing (computation).

Intelligence, as capacity for problem-solving within an environment/context, can be seen as one of the features of cognition. It is found in all living organisms as they all possess cognition, from single cells to their complex structures constituting tissues, organs, and organisms in constant interaction with each other and with the environment.

Human intelligence is the object of most of the studies of intelligence. Often it is considered to be a~multidimensional phenomenon, that includes both classical problem-solving and decision-making ability (logical-mathematical reasoning), existential intelligence (ability to survive), visual-spatial, musical, bodily-kinesthetic, naturalist, linguistic, interpersonal (social), and intra-personal (ability of inner insight) intelligence. However, the question of cognition and intelligence in non-human animals and other organisms is still controversial in philosophy of mind, psychology and even in some parts of cognitive science
%\label{ref:RNDF2iHJypmxo}(Ball, 2022).
\parencite[][]{ball_book_2022}.%


\section*{Cognition and intelligence on different levels of organization of life---embodied, embedded, enacted, and extended}
With the increasing insights into empirical details of processes and structures of cognition, it is emerging that human cognition and intelligence are based not only on activities of nervous system with neurons and glia cells, but equally importantly results from their interaction with non–neuronal subsystems including immune system and other somatic cells as well as the exchanges of the body with the environment. It comes as no surprise, as the nervous system is in a~close interaction with the rest of the body.

Human nervous system is made up of two types of cells: primary neurons and glial cells, and it is divided into two parts: the central nervous system (brain and spinal cord) and the peripheral nervous system (autonomic and somatic nervous systems). The nervous system controls and regulates the activities of organs and systems through neuron feedback, enabling the body to respond to environmental changes
%\label{ref:RNDMQODSfXl9p}(Biotechnology-Accegen, 2022).
\parencite[][]{biotechnology-accegen_nervous_2022}. %
 Through the embodiment, the nervous system also communicates with the external world, including other cognitive agents. The understanding that human cognition results from the activities of different types of cells, and not only nerve cells (neurons), is based on the new recognition of the existence of basal cognition/ minimal cognition / microorganismic cognition and intelligence. Unicellular organisms (single cells) have sensors and actuators and use chemical signaling and transfer of genetic information as a~basis for adaptation and learning 
%\label{ref:RNDQas6g1VQGC}(Baluška and Levin, 2016; Ng and Bassler, 2009; Witzany, 2011; Ben-Jacob, 2003; Ben-Jacob, Shapira and Tauber, 2006).
\parencites[][]{baluska_having_2016}[][]{ng_bacterial_2009}[][]{witzany_introduction_2011}[][]{ben-jacob_bacterial_2003}[][]{ben-jacob_seeking_2006}. %
Cognitive (sensory-based) and intelligent (problem-solving) processes are regulating the state of a~single cell which is a~building block of multicellular living organisms 
%\label{ref:RNDBkgIopiKne}(Manicka and Levin, 2019).
\parencite[][]{manicka_cognitive_2019}.%


Thus, recently the ideas of cognition and intelligence have increased in scope
%\label{ref:RNDG3aH2Szf40}(Dennett, 2017)
\parencite[][]{dennett_bacteria_2017} %
 with improved understanding of their underlying mechanisms---from the activity on the level of the human brain, to the processes on the somatic cell level. Single cell need not be a~part of a~human body to be seen as performing cognitive and intelligent behavior, it could be a~unicellular organism or a~constituent part of an animal or a~plant.

At the same time as new insights have been made into the nature of biological cognition, computational and robotic cognitive systems are being developed with various degrees of cognition and intelligence. Some functions of artificial intelligence surpass human capacities (such as processing parallelism, search, memory, precision, and correctness, and often also speed) while many other capacities are far below the human level, such as common-sense reasoning, or goal-directed agency in the sense of self-preservation and self-organization.

Understanding cognition and intelligence in nature on different levels of organization, because of their fundamentally biological mechanisms is only possible if we see it in the context of evolution. As in all of biology, ``nothing makes sense except for in the light of evolution''
%\label{ref:RNDIMpkF5w80b}(Dobzhansky, 1973),
\parencite[][]{dobzhansky_nothing_1973}, %
 and the cognition as a~process can only be understood in the light of evolution.

However, new \textit{abiotic approaches to cognition} assume that it is possible to construct cognitive agents from abiotic elements. Artificial (artifactual) intelligence is an attempt to produce intelligent behaviors akin to those shown by living beings (from the beginning specifically in humans) but implemented in non-living substrate. We can compare ``cognitive behavior'' of abiotic systems with the cognitive behavior of living organisms and see how close they are to each other.

\section*{The necessity of the Extended evolutionary synthesis (EES)}
Looking at the intelligence of a~living organism as a~result of information processing and embodied goal-directed behaviors on hierarchy of levels of organization, suggests necessity of understanding of the process of evolution in a~broader and more inclusive way than before, where biological agents are seen in their natural environments, from single cells to groups of organisms. A~scientific meeting organized in partnership with the British Academy by Denis Noble, Nancy Cartwright, Patrick Bateson, John Dupré and Kevin Laland presented and discussed those important \textit{New trends in evolutionary biology in biological, philosophical, and social science perspectives}
%\label{ref:RNDXEA2O22kAp}(Royal Society, 2016).
\parencite{royal_society_new_2016}.

That emerging view of evolution is called Extended Evolutionary Synthesis (EES), which is a~new interpretation of the theory of evolution based on the latest scientific knowledge about life and its changes, emphasizing fundamental mechanisms of constructive development and reciprocally causal nature between an organism and its environment
%\label{ref:RNDgkbSrVvLgK}(Schwab, Casasa and Moczek, 2019).
\parencite{schwab_reciprocally_2019}
More on Extended Evolutionary Synthesis can be found in
%\label{ref:RND9K6XPTJgQl}(Laland et al., 2015)
\parencite[][]{laland_extended_2015}%
, presenting EES and its structure, assumptions, and predictions, and
%\label{ref:RNDDNRedLSVaX}(Müller, 2017b; 2017a)
\parencites{muller_correction_2017}{muller_why_2017}
explaining why an extended evolutionary synthesis is necessary. Svensson 
%\label{ref:RNDHmkTlDpsk1}(2018)
\parencite*[][]{svensson_reciprocal_2018} %
 argues:

\myquote{
The Extended Evolutionary Synthesis (EES) will supposedly expand the scope of the Modern Synthesis (MS) and Standard Evolutionary Theory (SET), which has been characterized as gene-centered, relying primarily on natural selection and largely neglecting reciprocal causation. 

}
Evolution is a~result of interactions between natural agents, cells and their groups on variety of levels of organization
%\label{ref:RNDkYSc8QL3O8}(Jablonka and Lamb, 2014; Laland et al., 2015; Ginsburg and Jablonka, 2019),
\parencites[][]{jablonka_evolution_2014}[][]{laland_extended_2015}[][]{ginsburg_evolution_2019}, %
 as Jablonka and Lamb argue in their book \textit{Evolution in Four Dimensions}: \textit{Genetic, Epigenetic, Behavioral, and Symbolic Variation in the History of Life}. These dimensions can be found on different levels of organization of life.

In short, if we want to bring evolutionary theory in coherence with the advancement in other sciences, extended evolutionary synthesis is necessary.

\section*{Info-computational lens: agent-based natural information and computation}
We use an info-computational lens to approach phenomena of cognition and intelligence. A~framework of
%\label{ref:RND0tezG4JZl5}(Dodig-Crnkovic, 2017c)
\parencite{dodig-crnkovic_nature_2017}
 enables understanding of cognitive systems generated through self-structuring processes of morphological info-computation on the hierarchy of levels in nature from physics, to chemistry and biology, based on agent-centric embodied information and morphological computation. It means that we assume:

\begin{itemize}
\item 
computing nature paradigm, where nature is seen through the lens of information and computation as its dynamics, that is providing a~basis for unification of currently disparate understanding of natural, formal, technical, social and behavioral phenomena;
\item an observer-dependent, agent-based reality, that is reality for an agent for which cognition is a~result of relational info-computational processes;
\item 
computational interpretation of information dynamics in nature, where computation is physical (morphological) computation;
\end{itemize}
that enables us to:

\begin{itemize}
\item
avoid frequent misunderstandings of the inadequate abstract models of computation (as in old computationalism) and focus on embodied morphological computation in physical systems, especially cognitive ones such as living beings;
\item suggest the necessity of generalization of the models of computation beyond the traditional Turing machine model and acceptance of ``second generation'' models of computation capable of covering the whole range of phenomena from physics to cognition (Abramsky);
\item understand goal directed behaviors and complexification in living systems through the extended evolutionary synthesis.
\end{itemize}
The developments supporting info-computational approach, as a~variety of naturalism, are found in among others complexity theory, systems theory, theory of computation (natural computing, organic computing, unconventional computing), cognitive science, neuroscience, information physics, agent based models of social systems and information sciences, robotics (especially developmental robotics), bioinformatics and artificial life
%\label{ref:RNDEPyhvyjrLC}(Dodig-Crnkovic and Müller, 2011; Dodig-Crnkovic, 2017c),
\parencites[][]{dodig-crnkovic_dialogue_2011}[][]{dodig-crnkovic_nature_2017}, %
 as well as biosemiotics 
%\label{ref:RNDEfDB3UFzr6}(Sarosiek, 2021)
\parencite[][]{sarosiek_role_2021} %
 and Polak-Krzanowski's deanthropomorphized pancomputationalism 
%\label{ref:RNDOmoBkXb1MY}(Polak and Krzanowski, 2019).
\parencite[][]{polak_deanthropomorphized_2019}.%


Cognition of a~living organism is thus studied as a~network of networks of distributed information processing units on variety of levels of organization, from single cells to the whole body including the level of groups of organisms manifest as social cognition.

\section*{Natural cognition based on cells processing (computing) information---basal cognition in an extended evolutionary perspective}
Despite decades of research into the subject, there is still no agreement about where cognition is found in the living world
%\label{ref:RNDl1C34csuI6}(Ball, 2022).
\parencite[][]{ball_book_2022}. %
 Is a~nervous system needed? If so, why? If not, why not? A~new two-part theme issue of \textit{Phil Trans B} on the emerging field of ‘Basal Cognition', edited by Pamela Lyon, Fred Keijzer, Detlev Arendt and Michael Levin, explores these questions 
%\label{ref:RNDmadzdHb6j5}(Levin et al., 2021b; Lyon et al., 2021a).
\parencites[][]{levin_basal_2021}[][]{lyon_basal_2021}.%


Present increase of knowledge about cellular cognition and new gained details of complex goal-directed behaviors is nicely illustrated by the example of a~single-celled predator organism \textit{Lacrymaria olor} (``tears of a~swan'') hunting down another cell, often used by Michael Levin. \textit{Lacrymaria} has a~``neck'' a~``body'' and a~``mouth''. It beats the hair-like cilia around its ``head'' and extends its neck up to 8 times its body length, while chasing and finally swallowing another cell. It has no nervous system and no sensors that macroscopic living organisms typically use to chase their prey. How does it manage to identify, follow, and catch the prey? The mechanisms that enable \textit{Lacrymaria} to hunt down another cell, that goal-directedly activate cilia, take care of timing of ``mouth'' opening and closing are studied in
%\label{ref:RNDtWpihaoIiU}(Weiss, 2020).
\parencite[][]{weiss_single-celled_2020}. %
Likewise,
%\label{ref:RND1ouHm9Hh1n}(Coyle et al., 2019)
\parencite{coyle_coupled_2019}
describe how coupled active systems encode an emergent hunting behavior in \textit{Lacrymaria olor}. Even the work of 
%\label{ref:RNDtWFmPIiaIa}(Mearns et al., 2020)
\parencite[][]{mearns_deconstructing_2020} %
 analyzes its hunting behavior, revealing a~tightly coupled stimulus-response loop. Furthermore 
%\label{ref:RNDLrusy1RjT0}(Wlotzka and McCaskill, 1997)
\parencite[][]{wlotzka_molecular_1997} %
 argue that in this case, they observed behavior of a~molecular predator and its prey, through coupled isothermal amplification of nucleic acids. In short, research shows that a~goal-directed behavior of \textit{Lacrymaria olor}, is a~result of a~coupled stimulus-response loops. However, importantly, we do not know the meta-level mechanism which activates those loops and makes them goal directed.

Another microorganism under intense study for their goal-directed, efficient learning and adaptive behavior, which are of special interest because of their ability to cause diseases in other organisms, are bacteria. Eshel Ben Jacob have been studying bacterial colonies, their self-organization, complexification and adaptation, smartness, communication and linguistic communication (by chemical languages), social intelligence, natural information processing, and foundations of bacterial cognition
%\label{ref:RNDiu8o02ihyG}(Ben-Jacob, 1998; 2003; 2008; 2009; Ben-Jacob, Becker and Shapira, 2004; Ben-Jacob, Shapira and Tauber, 2006; 2011).
\parencites[][]{ben-jacob_bacterial_1998}[][]{ben-jacob_bacterial_2003}[][]{ben-jacob_social_2008}[][]{ben-jacob_learning_2009}[][]{ben-jacob_bacteria_2004}[][]{ben-jacob_seeking_2006}[][]{ben-jacob_smart_2011}. %
 Works 
%\label{ref:RNDdxHUs4ross}(Witzany, 2011; Schauder and Bassler, 2001; Waters and Bassler, 2005; Ng and Bassler, 2009)
\parencites[][]{witzany_introduction_2011}[][]{schauder_languages_2001}[][]{waters_quorum_2005}[][]{ng_bacterial_2009} %
 focus on communication (information exchange) mechanisms in bacteria, and especially \textit{quorum sensing}, where group of bacteria make a~majority-based decisions. Bacteria colonies and films display various multicellular behaviors, emitting, receiving, and processing a~large vocabulary of chemical symbols.

More about experimental methods for study of cell cognition can be found in the work of \textit{The cell cognition project}
%\label{ref:RNDdawchxSKWA}(Held et al., 2010).
\parencite[][]{held_cellcognition_2010}.%


From all the above evidence it is clear that unicellular organisms exhibit basal cognition and intelligence (problem-solving capacities). A~fundamental observation connecting this rudimentary biotic cognition and more complex anthropogenic (i.e., human-level, brain-based) cognition, is the following:

\myquote{
Cognitive operations we usually ascribe to brains---sensing, information processing, memory, valence, decision making, learning, anticipation, problem solving, generalization and goal directedness---are all observed in living forms that don't have brains or even neurons
%\label{ref:RND6vo9ZiyGzB}(Levin et al., 2021a).
\parencite[][]{lyon_basal_2021}.%


}
Similar arguments for biogenic nature of cognition have been presented by
%\label{ref:RNDmjYWflBJlx}(Lyon et al., 2021b; Yuste and Levin, 2021; Lyon and Kuchling, 2021).
\parencites[][]{levin_basal_2021}[][]{yuste_new_2021}[][]{lyon_basal_2021}.%


Our approach to information-processing mechanisms of cognition, unlike vast majority of artificial cognitive architectures targeting human-level cognition, focus on the development and evolution of the continuum of natural cognitive architectures, from basal cellular architecture up, as studied by
%\label{ref:RNDKd5AVQdO3d}(Lyon et al., 2021b)
\parencite[][]{levin_basal_2021} %
 and already identified by 
%\label{ref:RNDuRG37YOd8L}(Sloman, 1984).
\parencite[][]{sloman_structure_1984}.%


The connection between high-level and basal cognition is visible in the role of ion channels and neurotransmitters, studied in nervous cells, but also present in ordinary somatic cells:

\myquote{
We have previously argued that the deep evolutionary conservation of ion channel and neurotransmitter mechanisms highlights a~fundamental isomorphism between developmental and behavioral processes. Consistent with this, membrane excitability has been suggested to be the ancestral basis for psychology [...]. Thus, it is likely that the cognitive capacities of advanced brains lie on a~continuum with, and evolve from, much simpler computational processes that are widely conserved at both the functional and mechanism (molecular) levels.

The information processing and spatio–temporal integration needed to construct and regenerate complex bodies arises from the capabilities of single cells, which evolution exapted and scaled up as behavioral repertoires of complex nervous systems that underlie familiar examples of Selves
%\label{ref:RNDoMYcqHqs0S}(Levin, 2019).
\parencite[][]{fields_somatic_2019}.%


}
This biogenic nature of cognition makes it necessary to recognize all living forms, and not only those with nervous systems
%\label{ref:RNDZ7QWZmsJ5b}(Piccinini, 2020),
\parencite[][]{piccinini_neurocognitive_2020}, %
 or what is even more frequent only humans, as cognitive systems.

As for the driving mechanisms behind this complexification process in living/cognitive systems,
%\label{ref:RNDueX72UEzt5}(Fields et al., 2022, pp.1–2)
\parencite[][pp.1–2]{fields_free_2022} %
 describe how The Free Energy Principle of Karl Friston can drive neuromorphic development in the fully-general quantum-computational framework of topological quantum neural networks:

\myquote{
We show how any system with morphological degrees of freedom and locally limited free energy will, under the constraints of the free energy principle, evolve toward a~neuromorphic morphology that supports hierarchical computations in which each ``level'' of the hierarchy enacts a~coarse-graining of its inputs, and dually a~fine-graining of its outputs. Such hierarchies occur throughout biology, from the architectures of intracellular signal transduction pathways to the large-scale organization of perception and action cycles in the mammalian brain.

}
Biogenic approach is useful not only for understanding of cognition and intelligence and their evolution in living nature, but also for engineering of artificial systems that need certain level of intelligence, not necessarily the human level, such as nano-bots
%\label{ref:RNDsddBTIvFJ3}(Kriegman et al., 2021)
\parencite[][]{kriegman_kinematic_2021} %
 or different elements of IoT (Internet of Things).

\section*{Cognitive computing and AI---still anthropogenic}
Inspired by the behaviors produced by anthropogenic cognition,
%\label{ref:RNDzuzPZCdg8I}(Modha et al., 2011),
\parencite{modha_cognitive_2011}
the field of cognitive computing is exploring biomimetic approaches to cognition in abiotic systems 
%\label{ref:RNDExD86ynzYn}(Gudivada et al., 2019)
\parencite{gudivada_cognitive_2019}
studying cognitive computing systems, their potential and possible futures. In the application domain, e.g. IBM had a~cognitive computing project called Systems of Neuromorphic Adaptive Plastic Scalable Electronics (SyNAPSE)
%\label{ref:RNDbcK3pjmkp1}(Srinivasa and Cruz-Albrecht, 2012).
\parencite{srinivasa_neuromorphic_2012}.

The quest for intelligent machines ultimately requires new breakthroughs in computer architecture, theory of computation, computational neuroscience, supercomputing, cognitive science, and related fields orchestrated in a~coherent, unified effort.

Cognitive computing, AI and cognitive robotics present attempts to construct abiotic systems exhibiting cognitive characteristics of biotic systems. As a~rule, they assume human-level intelligence and human-level cognition, even though biogenic approaches would bring huge benefits. When we acknowledge that cognition in living nature comes in degrees, it is more meaningful to talk about cognition of artifacts, even though the role of cognitive capacities for an artefact is not to assure its continuing existence (unlike in cognition = life
%\label{ref:RNDKXwLubiX3M}(Stewart, 1996),
\parencite[][]{stewart_cognition_1996}, %
 which gives the evolutionary role to cognition in biotic systems). The difference is that cognitive artifacts are constructed to pursue human goals, not their own intrinsic ones.

\section*{Cognition at different levels of organization of a~living organism---from cells up}
Traditional anthropogenic approach to cognition
%\label{ref:RNDtzOKImtDnD}(Markram, 2012)
\parencite[][]{markram_human_2012} %
 is looking at cognition and intelligence in humans as the only natural cognitive agents.

Biogenic approaches on the other hand broaden the domain, seeing cognition as an ability of all living organisms
%\label{ref:RNDiBBkzuUqVI}(Maturana, 1970; Maturana and Varela, 1980; Stewart, 1996).
\parencites{maturana_biology_1970}{maturana_autopoiesis_1980}{stewart_cognition_1996}

More specifically, Maturana and Varela argue:

\myquote{
A~cognitive system is a~system whose organization defines a~domain of interactions in which it can act with relevance to the maintenance of itself, and the process of cognition is the actual (inductive) acting or behaving in this domain. \textit{Living systems are cognitive systems and living as a~process is a~process of cognition.} This statement is valid for all organisms, with and without a~nervous system
%\label{ref:RNDFN7IyFvYBJ}(Maturana and Varela, 1980, p.13; cf. Maturana and Varela 1992).
\parencites[][p.13]{maturana_autopoiesis_1980}[cf.][]{maturana_tree_1992}.%


}
Cognition is thus a~capacity possessed in different forms and degrees of complexity by every living organism. It is entirety of processes going on in an organism that keeps it alive, and present as a~distinct agent in the world. A~single cell while alive constantly cognizes, that is registers inputs from the world and its own body, ensures its own continuous existence through metabolism and food hunting while avoiding dangers that could cause its disintegration or damage, at the same time adapting its own morphology to the environmental constraints. The entirety of physico-chemical processes depends on the morphology of the organism, where morphology is meant as the form and structure. Work of Marijuán, Navarro and del Moral
%\label{ref:RNDJj50FSnmiv}(2010)
\parencite*{marijuan_prokaryotic_2010}
presents a~study of prokaryotic intelligence and its strategies for sensing the environment.

\section*{Multicellularity}
Unicellular organisms such as bacteria communicate and build swarms or films with far more advanced capabilities compared to individual organisms, through social (distributed) cognition.

In general, groups of smaller organisms (cells) in nature cluster into bigger ones (multicellular assemblies) with differentiated control mechanisms from the cell level to the tissue, organ, organism and groups of organisms, and this layered organization provides information processing benefits.

Examining the origin of multicellularity
%\label{ref:RNDZpvgrqhIX6}(Levin, 2019)
\parencite[][]{fields_somatic_2019} %
 investigates the computational boundary of a~``self'' and argues that it is bioelectricity that drives multicellularity and scale-free cognition. According to 
%\label{ref:RND5f922Es8X0}(Fields and Levin, 2019),
\parencite[][]{fields_somatic_2019}, %
 somatic multicellularity presents a~satisficing solution to the prediction-error minimization problem for single cells. From the point of view of information, 
%\label{ref:RNDMuA4JYwLre}(Colizzi, Vroomans and Merks, 2020)
\parencite[][]{colizzi_evolution_2020} %
 argue that evolution of multicellularity results from a~collective integration of spatial information, while 
%\label{ref:RNDuX5SnNWo0G}(McMillen, Walker and Levin, 2022)
\parencite[][]{mcmillen_information_2022} %
 show how to use Shannon information theory 
%\label{ref:RNDV2KqJQBm6J}(Shannon, 1948)
\parencite[][]{shannon_mathematical_1948} %
 as a~tool for integration of biophysical signaling modules. By mapping information flow between cells and pathways, researchers show that information theory supports systems-level view of biological phenomena where molecular reductionism does not work well.

\section*{Computationalism is not what it used to be …}
… that is the thesis that human cognition and intelligence are Turing machines
%\label{ref:RNDBnyqov3DdZ}(Scheutz, 2002).
\parencite[][]{scheutz_computationalism_2002}. %
 Unlike classical computationalism based on symbol manipulation and Turing Machine model, modern computationalism for modelling of cognitive processes requires new models of computation.

Turing Machine is an abstract logical model of computation equivalent to an algorithm, and it may be used for description of elementary sequential processes in living organisms. However, complex networked physical processes with temporal and other physical resource constraints cannot be adequately modelled as series of sequential logical operations (Turing machines). As Leslie Valiant
%\label{ref:RNDG2shdHw8MQ}(2013)
\parencite*[][]{valiant_probably_2013} %
 succinctly puts it:

\myquote{
We need computational models for the basic characteristics of life, such as the ability to differentiate and synthesize information, make a~choice, adapt, evolve, and learn in an unpredictable world. That requires computational mechanisms and models which are not ``certainly, exactly correct'' and predefined as Turing machine, but, instead, ``probably approximately correct'' (PAC).

}
Computational approaches that are capable of modelling adaptation, evolution and learning are found in the field of natural computation and computing nature
%\label{ref:RNDpwVgK7YJza}(Dodig-Crnkovic, 2014a).
\parencite[][]{dodig-crnkovic_modeling_2014}.%


\section*{Computing, the fourth scientific domain}
Info-computational approach incorporates our best current scientific knowledge about the processes in nature, translating them into language of natural information and computation.

The aim of this approach to cognition is to increase understanding of cognitive capacities in diverse types of agents, biological and synthetic, including their ability of learning, and learning to learn (meta-learning)
%\label{ref:RNDyYnQtnzvai}(Dodig-Crnkovic, 2020)
\parencite{dodig-crnkovic_natural_2020}
as well as their communication and mutual interactions. According to 
%\label{ref:RNDZKnHOIMKcu}(Denning, 2007),
\parencite{denning_computing_2007},
computing can be seen as a~natural science. Even more than that, we are witnessing the emergence of a~new computing science
%\label{ref:RND57Sj0r4uIO}(Denning, 2010)
\parencite{denning_computing_2010}
, connecting natural and formal sciences, adding the dimension of real time and physical constraints to logic and mathematics. As Rosenbloom argues, ``Computing may be the fourth great domain of science along with the physical, life and social sciences''
%\label{ref:RNDXq2ytez9Mf}(Rosenbloom, 2015).
\parencite{rosenbloom_computing_2015}.
In that new broader, emerging computing science, the Turing Model of computation is a~proper subset.

\section*{Computing nature and nature inspired computation. Self-generating systems}
Complex biological systems must be modeled as self-referential, self-organizing ``component-systems''
%\label{ref:RNDRVcipyZLh6}(Kampis, 1991)
\parencite[][]{kampis_self-modifying_1991} %
 which are self-generating and whose behavior, though computational in a~general sense, goes far beyond Turing machine model. Georg Kampis studied the behavior of self-modifying systems in biology and cognitive science as a~basis for a~new framework for dynamics, information, computation, and complexity:

\myquote{
a~component system is a~computer which, when executing its operations (software) builds a~new hardware. [... W]e have a~computer that re-wires itself in a~hardware-software interplay: the hardware defines the software, and the software defines new hardware. Then the circle starts again
%\label{ref:RNDWO7dKBBpPU}(Kampis, 1991, p.223).
\parencite[][p.223]{kampis_self-modifying_1991}.%
}

Similar position is presented in
%\label{ref:RND4Xm5NrJuqX}(Dodig-Crnkovic, 2011)
\parencite{dodig-crnkovic_dialogue_2011}
connecting models of computation from the formal sequential logical machine Turing model to the physical (morphological) concurrent natural computation.

\section*{Evolution as generative mechanism for increasingly complex cognitive systems}
New insights about cognition and its evolution and development in nature, from cellular to human cognition can be modelled as natural information processing/ natural computation/ morphological computation. In the info-computational approach, evolution in the sense of Extended evolutionary synthesis is a~result of interactions between natural agents, cells, and their groups.

Evolution provides generative mechanisms for the emergence of more and more competent living organisms, with increasingly complex natural cognition and intelligence, and those mechanisms can be used as a~template for the design and construction of their artifactual, computational counterparts.

\section*{Learning from biogenic computing}
The concept of biological computation posits that living organisms process information and thus perform computations, and that ideas of information and computation are the key to understanding, modeling, simulation, and control of biological systems. See
%\label{ref:RNDtiZRkCil5t}(Mitchell, 2012)
\parencite[][]{mitchell_biological_2012} %
 for the exposition of the concept of biological computation, and 
%\label{ref:RNDYyDKI20AfU}(Dodig-Crnkovic, 2022)
\parencite[][]{dodig-crnkovic_cognitive_2022} %
 for presentation of cognitive architectures based on natural infocomputation. Cognition as a~result of information processing in living agent's morphology, with species-specific cognition and intelligence is described in 
%\label{ref:RNDtR8OeVs1Xp}(Dodig-Crnkovic, 2021).
\parencite[][]{dodig-crnkovic_cognition_2021}.%


One of important characteristics of natural computing is its computational efficiency which is becoming increasingly important in the world with pervasive computing and concurrent global warming. The Turing Machine model of computation is not resource-aware, unlike living systems which are constantly optimizing their use of natural resources. Therefore, in the biomimetic approach to cognitive architectures designers are learning from nature how to compute more resource efficiently. Mutual learning between computing, cognitive sciences and neurosciences
%\label{ref:RNDBAH1u9dFp3}(Rozenberg and Kari, 2008)
\parencite[][]{rozenberg_many_2008} %
 leads to improved understanding of how cognition works and develops in nature, and how we can simulate, emulate, and engineer abiotic cognition and intelligence with the properties close to the biotic one.

\section*{Morphological computation connecting body, brain, and environment in robotics }
The research performed in the diverse fields of soft robotics / self-assembly systems and molecular robotics / self-assembly systems at all scales / embodied robotics / reservoir computing / real neural systems / systems medicine / functional architecture / organization / process management / computation based on spatio-temporal dynamics/ information theoretical approach to embodiment mechatronics~/ amorphous computing / molecular computing -- all connect body, control (``brain'') and environment.

In robotics, a~brain and body that researchers learn from, sometimes belongs to an octopus, which unlike typical robots has soft body that presents substantially different possibilities from rigid bodies of conventional robots.

Pfeifer and Bongard
%\label{ref:RNDZEG9lmZcOs}(2006)
\parencite*[][]{pfeifer_how_2006} %
 were among the first to present a~new view of embodied intelligence, arguing that the body shapes the way we think, looking in the first place from the anthropocentric perspective, but the approach applies equally well to biocentric view of cognition. In biologically inspired robotics, embodiment and self-organization are driving forces of evolving intelligence 
%\label{ref:RNDf851uu0IMn}(Pfeifer, Lungarella and Iida, 2007).
\parencite[][]{pfeifer_self-organization_2007}. %
 They are best understood in terms of morphological computation 
%\label{ref:RNDd417ZMqueA}(Pfeifer and Iida, 2005; Hauser, Füchslin and Pfeifer, 2014).
\parencites[][]{pfeifer_morphological_2005}[][]{hauser_opinions_2014}.%
\enlargethispage{1\baselineskip}


The essential property of morphological computation is that it is defined on a~structure of nodes (agents) that exchange (communicate) information. It is thus applied not only in robotics, but generalized to other physical information-processing systems, including living beings
%\label{ref:RNDUmZTsrN545}(Dodig-Crnkovic, 2013b; 2017b; 2018).
\parencites[][]{dodig-crnkovic_development_2013}[][]{dodig-crnkovic_morphologically_2017}[][]{dodig-crnkovic_cognition_2018}.%


\section*{Computing Nature and Natural Computation}
In his article ``Epistemology as Information Theory'', Greg Chaitin argues that knowledge should be studied as a~result of information processes, thus turning epistemology into study of information:

\myquote{
And how about the entire universe, can it be considered to be a~computer? Yes, it certainly can, it is constantly computing its future state from its current state, it's constantly computing its own time-evolution! And as I~believe Tom Toffoli pointed out, actual computers like your PC just hitch a~ride on this universal computation!
%\label{ref:RNDh2Lf1AamiY}(Chaitin, 2007, p.13)
\parencite[][p.13]{chaitin_epistemology_2007}%


}
David Deutsch in his article ``What is Computation? (How) Does Nature Compute?'' contributes with the similar position in the book ``A Computable Universe'' by Hector Zenil
%\label{ref:RND1vv5FDaP65}(2012).
\parencite*[][]{zenil_computable_2012}.%


Starting from the above ideas,
%\label{ref:RNDXDRXw0g64D}(Dodig-Crnkovic, 2007)
\parencite[][]{dodig-crnkovic_epistemology_2007} %
 proposes that epistemology can be naturalized through the info-computational approach to knowledge generation. The computing nature framework (naturalist computationalism) makes it possible to describe all cognizing agents (living organisms and artificial cognitive systems) as informational structures with computational dynamics 
%\label{ref:RNDKrdNppjXiv}(Dodig-Crnkovic and Burgin, 2011; Dodig-Crnkovic, 2013a; 2014b; 2017a; Dodig-Crnkovic and Giovagnoli, 2013; 2017).
\parencites[][]{dodig-crnkovic_dialogue_2011}[][]{dodig-crnkovic_computing_2013}[][]{dodig-crnkovic_modeling_2014}[][]{dodig-crnkovic_computational_2017}[][]{dodig-crnkovic_computing_2013}[][]{dodig-crnkovic_computational_2017}. %
 Morphological computation in this framework is a~process of creation of new informational structures, as it appears in nature, living as non-living. It is a~process of morphogenesis, which in biological systems is driven by development and evolution 
%\label{ref:RNDBhnHdYgrNx}(Dodig-Crnkovic, 2013b; 2017b; 2018).
\parencites[][]{dodig-crnkovic_development_2013}[][]{dodig-crnkovic_morphologically_2017}[][]{dodig-crnkovic_cognition_2018}.%


It is worth noting that research on ``computing nature'' focuses on how physical/ natural/ morphological processes can be interpreted as computation and used to compute, while research on ``computable universe'' asks the question if we can compute (with our current theories of computing) what we observe as the universe---two different research programs.

\section*{Conclusions}
New insights from complexity theory, systems theory, theory of computation (natural computing, organic computing, unconventional computing), cognitive science, neuroscience, information physics, agent based models of social systems and information sciences, robotics, as well as bioinformatics and artificial life call for updates in our understanding of cognition and intelligence
%\label{ref:RNDgZiYyO7wHY}(Dodig-Crnkovic and Müller, 2011; Dodig-Crnkovic, 2017c).
\parencites{dodig-crnkovic_dialogue_2011}{dodig-crnkovic_nature_2017}.

Traditionally, in the fields of cognitive science, philosophy of mind, cognitive computing and artificial intelligence, cognition and intelligence are assumed to be the abilities of humans. They are described in terms of concepts such as mind, thought, reasoning, logic, etc. However, new understanding of the goal-directed, learning, and adaptive behaviors of all living organisms, from all five kingdoms of life---animal, plant, fungi, protist and monera, from single celled to multi-cellular organisms and their ecologies, all possess level of cognition and intelligence which increases with the complexity of the system.

In this article we present a~common framework of info-computation, where computation is physical/morphological computation providing unified approach to anthropogenic, biogenic, and abiotic cognition. The advantage of info-computational approach is that it enables learning of mechanisms of those three types of cognition and intelligence. It also connects different levels of organization as observed in nature.

Cognition and intelligence, coming from the simplest to the most complex in a~continuum of natural systems can be source of inspiration for the design and construction of artificial cognitive systems with varying degrees/levels of intelligence, from nano-bots to autonomous cars and android robots.


\end{artengenv}