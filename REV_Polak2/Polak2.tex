\begin{newrevengenv}{Paweł Polak}
	{Beyond epistemic concepts of information: The case of ontological information as philosophy in science}
	{Beyond epistemic concepts of information\ldots}
	{Beyond epistemic concepts of information: The case of ontological information as philosophy in science}
	{Pontifical University of John Paul II in Krakow}
	{Roman Krzanowski, \textit{Ontological information: information in the physical world, series: World Scientific series in information studies}, 13, Hackensack, New Jersey: World Scientific 2022, pp. xii+264.}
	
	
	
%\begin{document}
%Beyond epistemic concepts of information: The case of ontological information as philosophy in science
%
%
%Roman Krzanowski, \textit{Ontological information: information in the physical world, series: World Scientific series in information studies}, 13, Hackensack, New Jersey: World Scientific 2022, pp. xii+264.

The concept of information plays and important role in science and philosophy, as well as in everyday life, such that it is now hard to imagine that this concept has been only adopted in the late 1940s. Many scientists find it even harder to believe that the concept of information can involve anything other than communication processes, and this is almost certainly due to Claude Shannon's Theory of Communication (TOC)
%\label{ref:RNDpOqVZWFMX0}(Shannon, 1949)
\parencite*[][]{shannon_mathematical_1949} %
 that entered the canon of unquestionable modern scientific knowledge. Unquestionably accepting Shannon's concept of a~measure of information entropy as the definition of information encourages a~scholar to treat TOC as scientific dogma.\footnote{ Shannon himself tried to warn against abusing his theory of communication (Shannon, 1956), though apparently unsuccessfully. He called against using the theory as a~source of hypotheses in other scientific disciplines. However, the opposite has happened—information metaphors have become unquestionable theoretical core of many modern concepts. This fact should not come as a~surprise, because if few people read the original work contenting themselves only with its processed results, it is difficult to suppose that anyone outside a~handful of specialists in the history of computing read Shannon's critical remarks very rarely cited in the literature.} However, studying the original work of Shannon and Weaver 
%\label{ref:RNDBAl2hvYeRw}(1964, p.3),
\parencite*[][p.3]{shannon_mathematical_1964}, %
 we realize that Shannon was primarily interested in communication engineering of digital signals (signal recovery, noise, optimal coding of signal), and the concepts of information and entropy have been borrowed by him from works of physicists like Ludwig Boltzmann, Leo Szilard, and John von Neumann.

After almost 80 years of being around, information is an elusive concept with manifold meanings; Krzanowski
%\label{ref:RNDa3uGRHNKEi}(2022)
\parencite*[][]{krzanowski_ontological_2022} %
 referred to more than 300 definitions of information, As information plays such an important role in contemporary societies, technology and science, it seems only logical that the efforts to clarify the meaning of information should never be abandoned. And this is precisely what Krzanowski's book is about.

However, while most of the published works on information see information through Shannon's lenses the focus of Roman Krzanowski's book
%\label{ref:RNDj1wWmH95Df}(Krzanowski, 2022)
\parencite[][]{krzanowski_ontological_2022} %
 is physical information. i.e., information that is not associated with knowledge or communication (Shannon's legacy), and that it is a~part of nature as other physical phenomena are; the conceptualization of information has not been widely accepted by the scientific community.

The first significant step was to distinguish between concepts of epistemic information (as found in Shannon's theory) from theories about ontological information. The author's second step was rather than constructing a~concept of information \textit{a~priori}, to proceed in the spirit of the Kraków school of philosophy in science
%\label{ref:RND6l6ZUmsrmK}(Heller, 2019; Polak, 2019; Polak and Trombik, 2022).
\parencites[][]{heller_how_2019}[][]{polak_philosophy_2019}[][]{polak_krakow_2022}. %
 Krzanowski searches for the meaning of ontological information g~attributed to it by reserachers and tries to understand the philosophical basis for using such concept. This is no coincidence, because in the pages of the associated journal ``Philosophical Problems in Science/Zagadnienia Filozoficzne w~Nauce'' (ZFN), the problem of information in science has been discussed from the very beginning to the current day 
%\label{ref:RNDqB1suQWgaa}(Turek, 1978; 1981; Krzanowski, 2017; 2020).
\parencites*[][]{turek_filozoficzne_1978}[][]{turek_rozwazania_1981}[][]{krzanowski_minimal_2017}[][]{krzanowski_why_2020}.%


The book is divided over seven chapters. The first chapter introduces the several definitions of information, often contradictory, demonstrating difficulty in accurately capturing the essence of this concept. Throughout the chapter the author gradually builds the conceptual base for his thesis and carefully justifies all his decisions. Of course, it is possible to disagree with Krzanowski on many issues, but one must admit that he tries to be very consistent, meaning that the deliberations as a~whole constitute a~valuable analysis of the concept of information. Even if one disagrees with the author's detailed claims, one would still concede that this book takes on an intriguing intellectual challenge and makes a~significant contribution to organizing and illuminating the discussion around the concept of information.

At a~time when authors mainly value their own originality, the work of Krzanowski has the characteristics of the best classical philosophy, which built its solutions on critical struggles with the heritage of tradition. It undoubtedly contributes to modern analytic philosophy, but the author's approach is to not simply copy contemporary models but instead creatively draw from various traditions, including Polish analytic philosophy. Although the author is far removed from the theses of Aristotelianism and scholasticism, his perfectly organized, methodical criticism and consistency and his precise argumentation is reminiscent of the style of Thomas Aquinas. More importantly, though, Krzanowski is not pragmatophobic, instead boldly pursuing solutions and seeking his own synthesis in the thicket of proposals. This method certainly sets this book apart from most works on the concept of information.

The crucial concept of ontological information is characterized as ``a physical phenomenon''
%\label{ref:RNDo7HyjxJ1Z2}(Krzanowski, 2022, p.6).
\parencite[][p.6]{krzanowski_ontological_2022}. %
 The author assumes that ``this information is perceived as a~structure, organization, or form of natural and artificial (artifacts) objects.'' He also defines this information as being objective and mind-independent while simultaneously clarifying all the concepts involved and trying to provide an argument for every claim. He also warns that ontological information is a~metaphysical concept and contributes to contemporary analytical metaphysics.

Krzanowski's analyses start with some intuitions about ontological information. He reconstructs ideas from dispersed quotations, much like how historians of philosophy deal with pre-Socratic philosophy. The methodology is similar because the concept of ontological information frequently manifests itself in the form of dispersed brief remarks.

The collection of these brief remarks by scientists is combined with a~careful interpretation and an attempt to reconstruct the philosophical intuitions they contain, the tasks that are the subject of the second chapter. This intriguing and inspiring journey passes through a~variety of ideas, culminating in the formulation of the eight main intuitions about ontological information that run through the scientific literature.

The third chapter analyzes the existing philosophical conceptions of ontological information, even though they do not usually refer to it using this term. We can find concepts coined by representatives of different disciplines from various countries, such as Carl von Weizsäcker, Krzysztof Turek, Stefan Mynarski, John Collier, Tom Stonier, Michał (Michael) Heller, Gordana Dodig Crnkovic, César Hildago, Thomas Nagel, Jacek Jadacki, and Anna Brożek. Krzanowski summarizes these concepts into 11 claims that are explained in detail
%\label{ref:RNDJPV06DPWXT}(Krzanowski, 2022, pp.86–93).
\parencite[][pp.86–93]{krzanowski_ontological_2022}.%


The aforementioned intuitions and claims serve as a~foundation for synthesizing the concept of ontological information in the fourth chapter, with Krzanowski ultimately reducing it to three claims:

\begin{itemize}
\item (EN) Information has no meaning, but meaning is derived from information by a~cognitive agent.
\item (PE) Information is a~physical phenomenon.
\item (FN) Information is responsible for the organization of the physical world.
\end{itemize}
This is then followed by two corollaries:

\begin{itemize}
\item (C1) Information is quanti[FB01?]able.
\item (C2) Changes in the organization of physical objects can be denoted as a~form of computation or information processing.
\end{itemize}
After the critical discussion, Krzanowski posits that these three properties and two corollaries are indispensable for the definition and understanding of the concept of ontological information.. This set of properties has a~hypothetical status, and this conceptualization is relative to actual science, so it is open to future changes together with the entirety of scientific knowledge.

The fifth chapter is devoted to broadly analyzing the problem of ontological and epistemological aspects of the concept of information. Krzanowski needs to adopt this perspective to further clarify the concept of ontological information. The analysis shows that while both concepts of information are required to account for the full spectrum of interpretations for information, ontological information appears to be more fundamental because it can serve as a~carrier of epistemic information.

In the following chapter, Krzanowski moves onto applications and interpretations of ontological information. He critically discusses the concept of an ``infon'' and data as basic concepts for defining information. The author claims that his conceptualization is more fundamental and better explains the source of epistemic information. Furthermore, Krzanowski attempts to resolve the dilemma of the contradictory abstract and concrete natures for information, and this is another original and inspiring aspect of his book. The example application of ontological information is Krzanowski's original concept of Minimal Information Structural Realism
%\label{ref:RNDX6YhDFBYAd}(first introduced in Krzanowski, 2017).
\parencite[first introduced in][]{krzanowski_minimal_2017}. %
 Also of interest is the consideration about possibly applying his conceptualization of ontological information to Popper's Three Worlds and Mark Burgin's General Theory of Information. Finally, Krzanowski applies Perzanowski's ontology to build an ontological foundation for the concept of ontological information.

The final chapter gathers together all the observations from the book, summarizes the key findings and conclusions, and brings up some selected criticisms of ontological information. (Krzanowski probably intentionally avoids the most common dogmatic critiques, regarding them as not being suitable for philosophical consideration.) Finally, through nine questions, the author reveals some perspectives for future research into ontological information. Each question opens up a~new field that could be a~subject of a~new study.

It is worth mentioning that the book has been carefully prepared from an editorial perspective. It has not, however, been spared from minor inaccuracies, such as the fact that Krzysztof Turek received his doctorate from the Pontifical Academy of Theology in Kraków, which only transformed into the Pontifical University of John Paul II in Kraków many years later. In addition, qualifying Jacek Jadacki as a~computer scientist
%\label{ref:RNDMJolJiNFG0}(Krzanowski, 2022, p.45)
\parencite[][p.45]{krzanowski_ontological_2022} %
 rather than a~philosopher and pianist is not only untrue—it is also inconsistent with the rest of the work. Nevertheless, these minor glitches do not significantly impair this important consideration of ontological information. Unfortunately, many more, albeit minor, errors can be found in the bibliography, especially in the Polish titles of works. This may hinder any search for the cited works in databases. Moreover, in some cases, works that have long been out of print, even five years ago, are still marked as being in print.

Assessing philosophical import of the book we may begin by noting that the book is relatively new, but published works have already used the ideas within it. Work that is worthy of mentioning here is that of the philosophy of information specialist Mark Burgin
%\label{ref:RNDIW4ajzgJLK}(Burgin and Mikkilineni, 2022).
\parencite[][]{burgin_is_2022}. %
 The ideas are also reflected in this issue of \textit{Philosophical Problems in Science} / ZFN 
%\label{ref:RNDq3Hhk28HzL}(Mścisławski, 2022).
\parencite[][]{mscislawski_is_2022}. %
 We should also emphasize here that the ideas presented in the reviewed book have resulted from the longer, critical reflection conducted by Krzanowski. This is especially true of the concepts of physical information and information, which were originally treated as being synonymous but have now been distinguished in the book, and this division has been well justified.

After reading the book, many questions can be raised, but to be fair to the author, they should be posed with great precision and care. There are certainly questions about whether the book finally resolves the problem of defining information or whether it finally explains the nature of information, but these would be misplaced questions. Indeed, it would be unacceptable for science to achieve these goals through \textit{a~priori} considerations, so if we adopt the scientific perspective, we must accept that we cannot provide definitive answers. Nevertheless, this does not mean that we must remain mute on the subject. On the contrary, we can say much about how the concept of information functions in modern science. Of course, Krzanowski's book only addresses the issue of ontological information, because philosophers have paid far too little attention to it. Indeed, the fame of Shannon's work on the theory of communication (often interpreted as the theory of information) has all too often led to an atrophy of criticism and a~limited vision for the nature of information. For this reason, the reviewed book makes a~valuable contribution to the discussion, and it not only reconstructs the concept of ontological information that is actually used in science but also critically evaluates it.

The book formulates a~set of properties of ontological information. This is the first attempt of its kind, and most importantly, it does not start from the author's arbitrary ideas but rather tries to deal with the thicket of intuitions and conceptualizations put forward in scholarly publications. The task is hard as scientists often hide their ignorance behind imprecise statements. After all, they are not professional philosophers, and in this task, which is secondary to their research, they may easily fall fowl of various errors or inaccuracies. This is perhaps why Einstein pointed out that one should pay attention to what scientists actually do rather than what they say about it. If this is indeed the case, then I~would like to propose an important area to develop research into the concept of ontological information, namely to investigate how it is actually used in scientific research.

Thus, it becomes necessary to go beyond the mere declarations and conceptualizations made by scholars and look at how the concept is used in explanatory structures and other aspects of research practice, at least if such contexts can be identified. Nevertheless, it should be noted that at the time of writing, such a~task was essentially impossible given the scarcity of studies using the concept of ontological information.

Of course, the author's assertion that the concept of ontological information is a~metaphysical concept should be borne in mind. Thus, the proposed line of research is also a~proposed case study of how metaphysics interacts with the sciences using the example of the modern concept of information. Such studies are also lacking, yet they could be of value to all philosophers of science who do not share the extreme anti-metaphysical position.

We can hope that this reviewed book and the study areas suggested for continuing research should have some impact on science. By stripping certain ideological trappings from the concept of information, broadening the perspectives (escaping Shannon's shadow) and making some necessary clarifications, scientists should certainly be able to develop new lines of research more effectively. Besides, the first harbingers of change have already appeared, such as the work flowing from the Kraków scientific community
%\label{ref:RND0MWvhNKRMw}(Bielecki and Schmittel, 2022),
\parencite[][]{bielecki_information_2022}, %
 which is based on the work of Krzanowski. Let us hope that other works using such a~``purified'' concept of ontological information in scientific practice will soon appear, because it will open up a~new yet important step in the philosophy of information, namely research into ontological information in scientific explanatory structures.

\paragraph{Acknowledgments}

The author thanks Łukasz Mścisławski for the in-depth discussions of the concept of information. His article
%\label{ref:RNDS1oz29763N}(Mścisławski, 2022)
\parencite[][]{mscislawski_is_2022} %
 published in this issue of ZFN is the first critical analysis of the concepts presented by R. Krzanowski; Łukasz Mścisławski's ideas significantly contributed to a~this review.

%Bibliography
%
%Bielecki, A. and Schmittel, M., 2022. The Information Encoded in Structures: Theory and Application to Molecular Cybernetics. \textit{Foundations of Science}, [online] 27(4), pp.1327–1345. https://doi.org/10.1007/s10699-022-09830-8.
%
%Burgin, M. and Mikkilineni, R., 2022. Is Information Physical and Does It Have Mass? \textit{Information}, [online] 13(11), p.540. https://doi.org/10.3390/info13110540.
%
%Heller, M., 2019. How is philosophy in science possible? \textit{Philosophical Problems in Science (Zagadnienia Filozoficzne w~Nauce}), [online] (66), pp.231–249. Available at: {\textless}https://zfn.edu.pl/index.php/zfn/article/view/482{\textgreater} [Accessed 6 October 2021].
%
%Krzanowski, R., 2017. Minimal Information Structural Realism. \textit{Philosophical Problems in Science (Zagadnienia Filozoficzne w~Nauce}), [online] (63), pp.59–75. Available at: {\textless}http://www.zfn.edu.pl/index.php/zfn/article/view/396{\textgreater} [Accessed 16 May 2018].
%
%Krzanowski, R., 2020. Why can information not be defined as being purely epistemic? \textit{Zagadnienia Filozoficzne w~Nauce (Philosophical Problems in Science}), [online] (68), pp.37–62. Available at: {\textless}https://zfn.edu.pl/index.php/zfn/article/view/494{\textgreater}.
%
%Krzanowski, R., 2022. \textit{Ontological Information: Information in the Physical World}. World Scientific series in information studies. [online] Hackensack, New Jersey: World Scientific Publishing Co. https://doi.org/10.1142/12601.
%
%Mścisławski, Ł., 2022. Is information something ontological, or physical or perhaps something else? Some remarks on R. Krzanowski approach to concept of information. \textit{Philosophical Problems in Science (Zagadnienia Filozoficzne w~Nauce}), (73), p.??
%
%Polak, P., 2019. Philosophy in science: A~name with a~long intellectual tradition. \textit{Philosophical Problems in Science (Zagadnienia Filozoficzne w~Nauce}), [online] (66), pp.251–270. Available at: {\textless}https://zfn.edu.pl/index.php/zfn/article/view/472{\textgreater} [Accessed 6 October 2021].
%
%Polak, P. and Trombik, K., 2022. The Kraków School of Philosophy in Science: Profiting from Two Traditions. \textit{Edukacja Filozoficzna}, (2(74)). https://doi.org/10.14394/edufil.2022.0023.
%
%Shannon, C.E., 1949. \textit{The mathematical theory of communication}. Urbana: University of Illinois Press.
%
%Shannon, C.E. and Weaver, W., 1964. \textit{The Mathematical Theory of Communication}. Illini Books. Urbana: University of Illinois Press.
%
%Turek, K., 1978. Filozoficzne aspekty pojęcia informacji. \textit{Philosophical Problems in Science (Zagadnienia Filozoficzne w~Nauce}), (1), pp.32–41.
%
%Turek, K., 1981. Rozważania o~pojęciu struktury. \textit{Philosophical Problems in Science (Zagadnienia Filozoficzne w~Nauce}), (3), pp.73–95.
%\end{document}

\vspace{15mm}%
{\subsubsectit{\hfill Abstract}}\\
{This review article discusses Andrzej Bielecki's book \textit{Models of Neurons and Perceptrons: Selected Problems and Challenges}, as published by Springer International Publishing. This work exemplifies ``philosophy in science'' by adopting a~broad, multidisciplinary perspective for the issues related to the simulation of neurons and neural networks, and the author has addressed many of the important philosophical assumptions that are entangled in this area of modeling. Bielecki also raises several important methodological issues about modeling. This book is recommended for any philosophers who wish to learn more about the current state of neural modeling and find inspiration for a~deeper philosophical reflection on the subject.}\par%
\vspace{2mm}%
{\subsubsectit{\hfill Keywords}}\\
{neuron modeling, sub-neuron modeling, computational modeling, analog computation, philosophy in science, philosophy of biology, philosophy of computing.}%



\end{newrevengenv}