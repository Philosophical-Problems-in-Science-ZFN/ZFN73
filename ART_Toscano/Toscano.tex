\begin{artengenv}{Javier Toscano}
	{But seriously: what do algorithms want? Implying collective intentionalities in algorithmic relays---a distributed cognition approach}
	{But seriously: what do algorithms want?\ldots}
	{But seriously: what do algorithms want? Implying collective\\intentionalities in algorithmic relays---a distributed cognition\\approach}
	{Center for Advanced Internet Studies, (CAIS), Bochum, Germany}
	{Describing an algorithm can provide a~formalization of a~specific process. However, different ways of conceptualizing algorithms foreground certain issues while obscuring others. This article attempts to define an algorithm in a~broad sense as a~cultural activity of key importance to make sense of socio-cognitive structures. It also attempts to develop a~sharper account on the interaction between humans and tools, symbols and technologies. Rather than human or machine-centered analyses, I~draw upon sociological and anthropological theories that underline social practices to propose expanding our understanding of an algorithm through the notion of ‘collective intentionalities.' To make this term clear, a~brief historical review is presented, followed by an argumentation on how to incorporate it in an integral perspective. The article responds to recent debates in critical algorithm studies about the significance of the term. It develops a~discussion along the lines of cognitive anthropology and the cognitive sciences, therefore advancing a~definition that is grounded in observed practices as well as in modeled descriptions. The benefit of this approach is that it encourages scholars to explore cognitive structures via archaeologies of technological assemblages, where intentionalities play a~defining role in understanding socio-structured practices and cognitive ecologies.
	}
	{algorithm studies, distributed cognition, collective intentionalities, socio-computing infrastructures, cognitive anthropology.}
	
	
%\begin{document}
%\title{But seriously: what do algorithms want?}
%\maketitle
%
%\title{Implying collective intentionalities in algorithmic relays---a distributed cognition approach}
%\maketitle
%
%\section*{Abstract}
%Describing an algorithm can provide a~formalization of a~specific process. However, different ways of conceptualizing algorithms foreground certain issues while obscuring others. This article attempts to define an algorithm in a~broad sense as a~cultural activity of key importance to make sense of socio-cognitive structures. It also attempts to develop a~sharper account on the interaction between humans and tools, symbols and technologies. Rather than human or machine-centered analyses, I~draw upon sociological and anthropological theories that underline social practices to propose expanding our understanding of an algorithm through the notion of ‘collective intentionalities'. To make this term clear, a~brief historical review is presented, followed by an argumentation on how to incorporate it in an integral perspective. The article responds to recent debates in critical algorithm studies about the significance of the term. It develops a~discussion along the lines of cognitive anthropology and the cognitive sciences, therefore advancing a~definition that is grounded in observed practices as well as in modeled descriptions. The benefit of this approach is that it encourages scholars to explore cognitive structures via archaeologies of technological assemblages, where intentionalities play a~defining role in understanding socio-structured practices and cognitive ecologies.
%
%\section*{Keywords}
%algorithm studies, distributed cognition, collective intentionalities, socio-computing infrastructures, cognitive anthropology

\section*{Initial Definitional Attempts}
\lettrine[loversize=0.13,lines=2,lraise=-0.03,nindent=0em,findent=0.2pt]%
{O}{}versimplified definitions of an algorithm are currently available and frequently used, but an algorithm is neither a~recipe nor a~rigidly constrained and procedural formulation. Limited conceptions that represent them as a~sort of entity or thing, a~series of steps that need to be applied, or a~simple technique that homogenizes a~process, lead to weak understandings of the deeper processes, transactions and dynamics that are at stake. Indeed, an algorithm can be a~problem-solving device, and this feature in itself can become a~point of entry to a~more complex analysis. After all, for engineers and computer scientists, ``an algorithm is an abstract, formalized description of a~computational procedure''
%\label{ref:RNDhginqouzXT}(Dourish, 2016, p.3).
\parencite[][p.3]{dourish_algorithms_2016}. %
 But even if sleek and apparently elegant, the problem with this definitional reduction is twofold.

On the one hand, it concentrates on processes that happen inside computational machines. This makes the description not only machine-centered, but also introduces a~misunderstanding. After all, articulating a~notion of code in the early days of computing history, pioneer logicians Newell, Simon and Shaw wrote in a~seminal paper that
\myquote{
the appropriate way to describe a~piece of problem solving behavior is in terms of a~program […]. Computers come into the picture only because they can, by appropriate programming, be induced to execute the same sequences of information processes that humans execute when they are solving problems
%\label{ref:RNDBmjqIEJ8RS}(Newell, Simon and Shaw, 1958, p.151).
\parencite[][p.151]{newell_elements_1958}. %
}
 In this sense, as programs, algorithms need not be thought of merely as machine drivers. And as we will see, getting rid of this idiosyncratic constraint would allow us to spot algorithms everywhere, as cultural artifacts 
%\label{ref:RNDhsaVRFFXMo}(Finn, 2017, p.15; Seaver, 2017).
\parencites[][p.15]{finn_what_2017}[][]{seaver_algorithms_2017}.%


On the other hand, it is also helpful to recall that many current and historical algorithms have not implied as part of their problem-solving process to attain their objective in a~neat, simple and efficient form. As a~matter of fact, some algorithms aim only at keeping a~solution in tension, without giving it away for everyone at every time (puzzles or riddles), while others simply produce contemplative outcomes, or even explicit nonsense (some art or literary pieces in the tradition of Dadaism, for instance). Solutions need not only be effective and efficient, they can also be creative, entertaining, experimental, contestatory, convoluted, tortuous or even purposely enigmatic or mistaken. Without reflecting on the diversity of possible outcomes and their consequences, computer scientists have usually produced equivalences between physical realities and formal symbolic systems, which have minimized the variety of ``solutions'' of the human world. Obviously this is not ``wrong''. It is what is expected from our computational machines under the dominant social normativity. But our understanding should not mistake an effect for a~cause. And of course, this does not bring us closer to a~precise definition of an algorithm in the broader sense that is implied here, even if it makes clearer the scope of the task.

If we take these initial considerations into account, we can see that, in order to look for a~definition of an algorithm that describes both what engineers do when they program a~computer, and what users do when they tinker and apply that program, or simply invent parallel procedures for the problem they attempt to solve through any other physical technology, we need to take a~different approach. First of all, we need to recognize that an algorithm is an attempt to bring something into the material world (an idea, a~calculation, a~previous experience). Clearly, this does not mean that every symbol will have a~physical manifestation, but that symbols are intermediaries, pieces that attempt to make a~translation between the ideal and the material. For as Lev Vygotsky
%\label{ref:RNDg8HO81pwku}(1978)
\parencite*[][]{vygotsky_tool_1978} %
 explains, what we conventionally call tools and what we conventionally call symbols are two aspects of the same phenomenon. According to him, mediation through tools could be seen as more outwardly oriented, while mediation through signs could be seen as more inwardly oriented, toward ``the self'', but both aspects emerge in every cultural artifact.

If we apply this notion of artifact mediation to our search, we can see that, whether through direct tools or indirect symbols, an algorithm implies an iterative interaction with technology, or in other words, a~practice of recursive intertwining between humans and the technologies they produce. Yet, for Vygotsky, an interaction with symbols or tools is not simply functional, in the sense in which a~subject manipulates an object at will to achieve a~task. Instead, cultural artifacts regulate interactions with one's environment and with oneself. But this is not innocuous: cultural mediation has a~recursive, bidirectional effect; mediated activity simultaneously modifies both the environment and the subject. Cultural mediation influences behaviors, synthetizes experiences from forbearers, and prepares children to acquire specific sets of accumulated memories, as knowledge
%\label{ref:RNDW8eWxTv3J8}(see here also Connerton, 1989).
\parencite[see here also][]{connerton_how_1989}. %
 This co-constitution is what Malafouris, along a~series of cognitive examinations, has termed as metaplasticity 
%\label{ref:RNDg6Z2NrdKR9}(Malafouris, 2010; 2013; 2015).
\parencites[][]{malafouris_metaplasticity_2010}[][]{malafouris_how_2013}[][]{malafouris_metaplasticity_2015}.%


In the end, cultural mediation---or the ability to think and operate through cultural artifacts---produces historical modes of thinking---i.e. ideologies---and styles of cognition that affect how we learn, think and represent our environment and ourselves. This is what can be described as the notion of distributed cognition
%\label{ref:RNDSg24DwdaBd}(Cole and Engeström, 1993; Gallagher, 2005; 2013).
\parencites[][]{salomon_cultural-historical_1993}[][]{gallagher_how_2005}[][]{gallagher_socially_2013}. %
 Laland et al. 
%\label{ref:RNDCkpURhADXA}(2000, p.177)
\parencite*[][p.177]{laland_niche_2000} %
 provide a~definition of this process: ``Distributed cognition means more than that cognitive processes are socially distributed across the members of a~group. It is a~broader conception that includes phenomena that emerge in social interactions as well as interactions between people and structure in their environments.'' The notion of distributed cognition has been a~common hypothesis in linguistics and psychology ever since the writings of Vygotsky were published and made available in different translations since the 1970s, but it is by no means the standard model. There are many critics that maintain that, even if aided through different tools, thinking happens basically inside the brain 
%\label{ref:RNDz69682DG3T}(Adams and Aizawa, 2008; 2010; Loh and Kanai, 2016),
\parencites[][]{adams_bounds_2008}[][]{menary_defending_2010}[][]{loh_how_2016}, %
 or they present situations in which thinking is affected from ``outside'' factors (which Clowes
 \parencite*[][]{clowes_screen_2019} %
 terms as ``the impact thesis''%
%\label{ref:RND8Ryfr6A8K0}(Clowes, 2019)
%\parencite[][]{clowes_screen_2019}%
). We will not deal here with those arguments, since they appear to have a~strong need for an essentialist form of conceptualizing cognition.

Moreover, since culture is for any notion of distributed cognition a~foundational concept, anthropologists have made major contributions to our understanding of both the implementation of culturally mediated forms of cognition and the various ways in which the heterogeneity of culture supports and requires the distribution of cognition. One of these anthropologists was Clifford Geertz. For Geertz, individuals submit themselves to governance by symbolically mediated programs for producing artifacts, organizing social life, or expressing emotions. In this recurring process that reaches every layer of an individual's life, humankind determines, if unwittingly, ``the culminating states of its own biological destiny''
%\label{ref:RNDn8EVwnZ89y}(Geertz, 1973, p.48).
\parencite[][p.48]{geertz_interpretation_1973}. %
 He states, in a~formulation that evokes the Vygotskian approach:

\myquote{
[S]ymbols are thus not mere expressions, instrumentalities, or correlates of our biological, psychological, and social existence; they are prerequisites of it. Without men, no culture, certainly; but equally, and more significantly, without culture, no men
%\label{ref:RNDKLhGzqGJ1x}(1973, p.49).
\parencite*[][p.49]{geertz_interpretation_1973}.%
}

Geertz's formulation found strong empirical evidence among theoretical biologists, for whom the connection between culture and biology implied more than a~simple correlation. As Laland et al.
%\label{ref:RNDOS0n99sbzG}(2000, p.131)
\parencite*[][p.131]{laland_niche_2000} %
 later would claim: ``cultural traits, such as the use of tools, weapons, fire, cooking, symbols, language, and trade, may have played important roles in driving hominid evolution in general and the evolution of the human brain in particular'' 
%\label{ref:RND7shgYMzA5p}(see also Dunbar, 1993; or Aiello and Wheeler, 1995).
\parencites[see also][]{dunbar_coevolution_1993}[or][]{aiello_expensive-tissue_1995}. %
 Nonetheless, when Geertz and other social scientists started confining everything under the domain of ``culture'', throughout the 1980s, the concept became too broad and lost its specific, explanatory power. As Nick Seaver 
%\label{ref:RNDJO9JEsgpIN}(2017, p.4)
\parencite*[][p.4]{seaver_algorithms_2017} %
 writes: ``Its implicit holism and homogenizing, essentialist tendencies seemed politically problematic and ill suited to the conflictual, changing shape of everyday life.'' As a~response, one of the most resourceful attempts in the social sciences to overcome the difficulties brought about by an all-encompassing concept---which was nonetheless useful as a~theoretical compass on a~structural level---was to turn to the study of practices and symbolic interactions 
%\label{ref:RNDtoIzySTaYH}(Bourdieu, 1972; Certeau, 1984; Blumer, 1986).
\parencites[][]{bourdieu_outline_1972}[][]{certeau_practice_1984}[][]{blumer_symbolic_1986}. %
 Consequently, many sociologists and anthropologists turned from a~vision of a~frame culture as a~unified domain, to the multiplication of sites and cultures, where they could study and map emerging symbolic orders, sometimes coordinated, sometimes conflicting, out of which to make sense of the different layers of social life. This approach left behind the deterministic tone of previous explanations, with their emphasis on rules, models and texts, and began focusing instead on strategies, interests, improvisations and interactional occurrences. Recovering this emphasis, and back to our line of inquiry, Seaver 
%\label{ref:RNDxmbiXf5hFC}(2017, p.5)
\parencite*[][p.5]{seaver_algorithms_2017} %
 provides a~description of an algorithm that is worth mentioning:
 \myquote{
 Like other aspects of culture, algorithms are enacted by practices which do not heed a~strong distinction between technical and non-technical concerns, but rather blend them together. In this view, algorithms are not singular technical objects that enter into many different cultural interactions, but are rather unstable objects, culturally enacted by the practices people use to engage with them.
 }
 
Seaver highlights the relational aspect of processes, enacted by practices rooted in cultural codes, therefore avoiding both a~subject-centered perspective as well as a~machine-centered view. In that sense, his definition is in line with a~number of interesting theories and methodologies that have emerged in sociology and science and technology studies over the past two decades, for example: actor-networks
%\label{ref:RNDKCObcDcGrt}(Callon, 1986; Latour, 1992; 2005),
\parencites[][]{law_elements_1986}[][]{bijker_where_1992}[][]{latour_reassembling_2005}, %
 sociotechnical ensembles 
%\label{ref:RNDxMte83gBXl}(Bijker, 1999),
\parencite[][]{bijker_bicycles_1999}, %
 object-centered socialities 
%\label{ref:RNDqe3WU3um8T}(Knorr Cetina, 1997),
\parencite[][]{knorr_cetina_sociality_1997}, %
 relational materialities 
%\label{ref:RNDyeyw2YVbeV}(Law, 2004),
\parencite[][]{law_after_2004}, %
 constitutive entanglements 
%\label{ref:RNDJ4mXcd66fz}(Orlikowski, 2007)
\parencite[][]{orlikowski_sociomaterial_2007} %
 or object-oriented ontology 
%\label{ref:RNDSDtw883D9O}(Harman, 2002; Bryant, 2010),
\parencites[][]{harman_tool-being_2002}[][]{bryant_onticologymanifesto_2010}, %
 as well as the approach of cognitive ecology 
%\label{ref:RNDBy4aR50ipy}(Hutchins, 2010)
\parencite[][]{hutchins_cognitive_2010} %
 and material engagement theory 
%\label{ref:RNDdpO41i1cNn}(Malafouris, 2005; 2013)
\parencites[][]{malafouris_cognitive_2005}[][]{malafouris_how_2013} %
 in the cognitive sciences. These theories challenge and transcend conventional distinctions between objects and subjects, as well as between social abstractions and material iterations. Furthermore, their particular value lies in their insistence on speaking of the social (e.g. culture) and the material (e.g. nature) in the same register, and on not resorting to a~limiting dualism that treats them as separate, even if interacting, phenomena.

Being an anthropologist, Seaver concentrates on the instabilities, the discontinuities, the confusions, the contradictions and the misunderstandings that enable different traditions and enrich human life. However, his view can be further explored, since it lacks a~reasonable explanation of how, despite being categorized as ``unstable objects'', algorithms may appear as robust, reliable and even intrinsically repeatable. In other words, how do procedural patterns are sustained, despite variance; how consistencies emerge to enable traditions; when are recurrences broken up and when are they maintained? These inquiries are relevant because algorithms are something more than people executing socially available recipes and tweaking them with a~personal taste. Algorithms are clusters of affordances and patterns that emerge in every process of \textit{recursive} intertwining between humans and technologies. In that sense, they could be seen as material or immaterial scripts that link mental states with both material procedures and technocultural resources, enacted as a~cultural practice to accomplish a~specific task (effectively or not). And yet, in this description, mental states need not pertain to a~single individual. Actually, if they would really belong to a~unique individual (someone looking for a~unique solution to his/her own problems, desires or needs) they would be socially illegible. But shouldn't this call for the inference of collective mental states? And what would that entail? The issue demands a~deeper inspection, and we will now turn to it.

\section*{The mental and the notion of collective intentionalities}
In order to inspect closer how humans and technologies interact through material or immaterial procedures linking mental states to real-world conditions, we need to acknowledge what we mean by mental states, and how they emerge as techno-cultural practices out of which specific patterns can be traced. This will require a~short detour to explain some basic conceptions, but by the end of this explanation we will have a~clearer landscape of the categories at stake.

A~mental state can best be delineated by the notion of intentionality. Intentionality is a~complex philosophical concept that emerged with Medieval Scholasticism through Medieval Islamic philosophy, but was later retaken and developed in phenomenological circles, starting from the 19\textsuperscript{th} century. Franz Brentano's work is usually set as a~point of departure for contemporary analyses. In his writings, intentionality is set as an attribute of an individual's mind, which adheres to mental contents, as opposed to attributes of the real world, such as extension and duration, which can be predicated of existing objects. Brentano takes on the discussion from St. Thomas Aquinas, who established that the object which is thought is intentionally in the thinking subject, the object which is loved in the person who loves, the object which is desired in the person desiring, etc. In that sense, intentionality is clearly something that can be predicated of inexistent phenomena, but which has an effect on our own conceptions, desires and beliefs. Brentano
%\label{ref:RNDqcysZzdwoS}(1995, p.68)
\parencite*[][p.68]{brentano_psychology_1995} %
 writes:

\myquote{
Every mental phenomenon is characterized by […] the intentional (or mental) inexistence of an object, and what we might call, though not wholly unambiguously, reference to a~content, direction toward an object (which is not to be understood here as meaning a~thing), or immanent objectivity. Every mental phenomenon includes something as object within itself […] We can, therefore, define mental phenomena by saying that they are those phenomena which contain an object intentionally within themselves.

}
At this point, intentionality was described as a~clear attribute of mental activity, independent of a~real world, but clearly related to it, and decisive to ascribe it meaning. This trait was important because it offered a~form of cognizing reality without relying on the Kantian formulation that attempted to align (individual) sensations and (social) concepts. In other words, it created a~model where things could be cognized beyond a~thick web of structured epistemological pre-conceptions. This is precisely what encouraged Husserl's enthusiasm, as inscribed in his motto ``Back to the things themselves!'' (\textit{Zurück zu den Sachen selbst!}). For as Merleau-Ponty
%\label{ref:RNDs2qs7wId98}(2005, p.xix)
\parencite*[][p.xix]{merleau-ponty_phenomenology_2005} %
 writes:
 \myquote{
 What distinguishes intentionality from the Kantian relation to a~possible object is that the unity of the world, before being posited by knowledge in a~specific act of identification, is ‘lived' as ready-made or already there.
 }

However, intentionality in this early stage also made a~clear difference between the inner, mental world, and the outer, objective reality. In that sense, it was still trapped in the fundamental dualism that characterized the positivist style of thinking in the late 19\textsuperscript{th} and early 20\textsuperscript{th} centuries. This dynamic has been sufficiently deconstructed, especially within the theories that were mentioned in the previous section, and there is no need to discuss it further. A~second problem is that this early notion of intentionality also posited a~very clearly delimited ``self'' for whom an intention (and communication of that intention) is transparent. The precise refutation of this point can be extensive, and it can also run through diverging lines, but for synthetic aims, we can resort back to the Vygotskian approach and understand the ``self'' as a~symbol and a~cultural artifact. Actually, both Vygotsky and a~contemporary anthropologist of him, G.H. Mead, worked along the lines of a~similar hypothesis, which has been termed the ``social genesis of the self''
%\label{ref:RNDoKclUAGbbL}(Glock, 1986),
\parencite[][]{glock_vygotsky_1986}, %
 and which implied both the process of internalisation (through education in the child) and the genesis of linguistic meaning. For Mead 
%\label{ref:RNDni2KOisKdU}(1972, p.164),
\parencite*[][p.164]{mead_mind_1972}, %
 for instance, ``[t]he process out of which the self arises is a~social process which implies interaction of individuals in the group, implies the preexistence of the group.'' Accordingly, he adds:

\myquote{
the self appears in experience essentially as a~``me'' with the organization of the community to which it belongs. This organization is, of course, expressed in the particular endowment and particular social situation of the individual […]. He is what he is in so far as he is a~member of this community, and the raw materials out of which this particular individual is born would not be a~self but for his relationship to others in the community of which he is a~part
%\label{ref:RNDT0MXT4rgdz}(Mead, 1972, p.200).
\parencite[][p.200]{mead_mind_1972}.%
}
Following the Vygotsky/Mead hypothesis, there cannot even be a~``direct'' connection between an individual and her experience, because this connection is mediated through language, by which a~``self'' appears as some type of thing. In other words, the emergence of a~``self'' is an effect, or a~functional construction, of a~subject that has learned how to enunciate and use the particle ``I'' under a~given set of socially-sanctioned, grammatical rules. This brings us to a~rather interesting situation on the cognitive side. For if the self is a~social construction, what is to be done with what we call ``the mental''? Is the link between both notions merely a~deficient attribution, or is it a~faulty causal connection? Mead describes the mental as an emergent phenomenon, which involves a~relationship to the character of things:

\myquote{
Those characters are in the things, and while the stimuli call out the response which is in one sense present in the organism, the responses are to things out there. The whole process is not a~mental product and you cannot put it inside of the brain. Mentality is that relationship of the organism to the situation which is mediated by sets of symbols
%\label{ref:RNDaP95DqbXRy}(Mead, 1972, pp.124–125).
\parencite[][pp.124–125]{mead_mind_1972}.%
}
This turns irrelevant the attribution of mentality to the self. On the same grounds, a~causal connection between them can only be inferred as inexistent. Instead, both are equally emergent effects of a~given symbolic mediation. Mead's description of the mental (that cognitive relationship of an organism to a~situation, mediated by symbols) is the backbone to the definition of an algorithm that was proposed on the previous section. It is also a~touchstone in the tradition of cognitive anthropology that has been associated with the idea of cognitive ecologies
%\label{ref:RNDcmyxI6a8WG}(Douglas, 1986; Lave, 1988; Connerton, 1989; Hutchins, 1995; 2010),
\parencites[][]{douglas_how_1986}[][]{lave_cognition_1988}[][]{connerton_how_1989}[][]{hutchins_cognition_1995}[][]{hutchins_cognitive_2010}, %
 as well as in traditions of cognitive sciences that inquire into models of an embodied, embedded, extended and/or an enactive social mind 
%\label{ref:RNDHOwURfIJuQ}(Clark, 1997; 2003; 2015; Clark and Chalmers, 1998; Gallagher, 2005; 2013; Gallagher and Miyahara, 2012).
\parencites[][]{clark_being_1997}[][]{clark_natural-born_2003}[][]{clark_what_2015}[][]{clark_extended_1998}[][]{gallagher_how_2005}[][]{gallagher_socially_2013}[][]{gallagher_neo-pragmatism_2012}. %
 Gallagher 
%\label{ref:RNDdhebNGSoIx}(2013, p.4),
\parencite*[][p.4]{gallagher_socially_2013}, %
 for instance, describes the mental in this way:

\myquote{
If we think of the mind not as a~repository of propositional attitudes and information, or in terms of internal belief-desire psychology, but as a~dynamic process involved in solving problems and controlling behavior and action---in dialectical, transformative relations with the environment---then we extend our cognitive reach by engaging with tools, technologies, but also with institutions. We create these institutions via our own (shared) mental processes, or we inherit them as products constituted in mental processes already accomplished by others.

}
Indeed, breaking the causal link between the mind and the self allow us to see the dense and emergent network of affordances and enactions that constitute cognitive phenomena. But how do intentionalities come back into the picture? For Brentano, intentionalities were so much as the mark of the mental, i.e. the defining quality of an in-existent, psychological phenomenon. But if the mind is not any more located in an inner, private world, should we just simply do without them? Quite the opposite. As a~matter of fact, intentionalities play a~stronger role within a~distributive cognition approach. But we need to refine the conceptual frame to see how this can be integrated into a~comprehensive explanation.

An intentionality is not a~purpose, nor a~design or an intention to do something, although the notions are closely related. Actions are intentional, for example, not only because there is a~will behind them, but also because they follow a~goal or a~project. If I~am hungry and I~do not have anything to eat at home, I~can go out to a~supermarket to buy groceries in order to cook, or to a~restaurant, or even to a~place where food is distributed if my economic means are limited. These, among others, are available modes of action, connected to material and technical functions, social behaviors and actionable symbolic networks. But we know that there used to be a~time when, if hungry, people could go out hunting or foraging, and the relevant social programs were there to support those activities. Intentionalities are attached then to historical norms, cultural repertoires, social habits, communal values, rituals and many other forms and forces that can be seen to shape an individual's action. For as Brandom
%\label{ref:RNDYwW0p4SJeW}(1994, p.61)
\parencite*[][p.61]{brandom_making_1994} %
 writes: 
 \myquote{
 only communities, not individuals, can be interpreted as having an original intentionality. [T]he practices that institute the sort of normative status characteristic of intentional states must be social practices.
 }
 In that sense, the social life of an individual consists in a~good deal in determining the appropriateness of her own desires and needs as these are articulated to the available social practices, or cultural programs, through inferential reasoning, practical adjustments and other means.

Now, even if at a~first look this explanation seems to restrain an individual's agency, by making her guide a~certain ``intended'' action through a~given catalogue of socially sanctioned paths, the picture that this model enables is in fact richer and more complex. In a~few words, a~strict functionalism does not apply
%\label{ref:RND4RkvJYvKWK}(Elster, 1983; Douglas, 1986, p.32ff).
\parencites[][]{elster_explaining_1983}[][p.32ff]{douglas_how_1986}. %
 As a~matter of fact, a~model like Malafouris' material engagement theory actually sustains that the distinctive forms of human agency emerge precisely in the practical space afforded by the interactions 
%\label{ref:RNDTsgW80JhWZ}(Malafouris, 2008; 2015).
\parencites[][]{malafouris_at_2008}[][]{malafouris_metaplasticity_2015}. %
 After all, an individual never ``acts'' in a~void either. And as Cooren et al 
%\label{ref:RND8KlhTAyYvS}(2006, p.11)
\parencite*[][p.11]{cooren_communication_2006} %
 write:
 \myquote{
 Agency is not a~‘capacity to act' to be defined a~priori. On the contrary, it is ‘the capacity to act' that is discovered when studying how worlds become constructed in a~certain way.
 }
 In that sense, intentionalities are sustained in social practices without losing their capacity for an individual's adaptation, expression and further innovation. And as such, they can be acknowledged as \textit{collective intentionalities}, fundamental pieces that connect an individual to a~larger collective, without necessarily turning them into a~deterministic setup. Collective intentionalities are in that sense something as action-able paths, through which an individual orients and articulates her actions with the resources and experiences of a~cultural community, i.e. a~community of practice.

Furthermore, collective intentionalities are so relevant that Tomasello
%\label{ref:RNDu6Ih0lNpVk}(2014)
\parencite*[][]{tomasello_natural_2014} %
 assigns to them, in an appealing hypothesis, a~definite role in the evolution of the species, since they allow coordination and cooperation to occur not only simultaneously, but also throughout generations. For this cognitive linguist, collective intentionalities comprise
 \myquote{
 not just symbolic and perspectival representations but conventional and ‘objective' representations; not just recursive inferences but self-reflective and reasoned inferences; and not just second-personal self-monitoring but normative self-governance based on the culture's norms of rationality 
%\label{ref:RNDmOPGUGVFHH}(Tomasello, 2014, p.6).
\parencite[][p.6]{tomasello_natural_2014}. %
}
 As such, they are the infrastructure of social life, underlying even culture and language through pre-linguistic aims and forces that acquire a~given shape. Developing over the foundations of collective intentionalities,
 \myquote{
 culture and language, as agent-neutral conventional phenomena […] provide another setting within which a~new form of human sociality can lead to a~new form of human thinking, specifically, objective reflective-normative thinking
%\label{ref:RNDY0mDZGZu4y}(Tomasello, 2014, p.141).
\parencite[][p.141]{tomasello_natural_2014}. %
}
 In that sense, collective intentionalities can be said to be the building blocks of human-symbol/tool interactions. But in the end, if collective intentionalities are not a~quality of the objective world---but rather its foundation---where are these to be seen, or how do they emerge and provide tangible samples for interactions to occur? We will tackle the issue in the following section.

\section*{Collective intentionalities and algorithms: heuristics and dynamics}
The notion of collective intentionality is only such if it retains one condition that was there since Brentano attempted a~definition: it is an attribute of a~mental state, i.e. a~mark of the mental. But we have seen that, in a~distributed cognition approach, the mental cannot be exclusively associated with a~self; it is rather an articulated web that links individuals to tools and symbols that have been pre-structured by a~collective, and are enacted through social practices. So we are presented with an empirical challenge: how to spot an intentionality if it is neither an objective nor a~subjective phenomenon in the classical sense? Collective intentionalities are usually ``hidden'' to the naked eye, sometimes they are by-products of repeated actions, much as a~trailing path in the woods which appears after years and years of different individuals walking through it, but sometimes they also stand out in oblique moves. In any case, they comprise the causal loops that run behind collective articulations (making up a~good deal of group identities, for example), and these can emerge as latencies, background or naturalized conventions
%\label{ref:RNDVdLRv9aQRl}(a specialized analysis in Chant, Hindriks and Preyer, 2014; in relation to this topic see also Toscano, 2024).
\parencites[a specialized analysis in][]{chant_individual_2014}[in relation to this topic see also][]{toscano_intentionalities_nodate}. %
 The only minimal assumption is that they stand in a~threshold, as that which allows community survival without demanding from individuals that they give up on their autonomy (even though the threshold is dynamic, and is not the same for a~child as for the elder, or throughout different knowledge capacities and hierarchies).

In the last years, cognitive scientists have developed different models to locate intentionalities via distinctive approaches. The neo-behaviourist Daniel Dennett, for instance, ascribes intentionality to observed rational behavior, and he describes the agent as someone 
\myquote{
who harbors beliefs and desires and other mental states that exhibit intentionality or ‘aboutness', and whose actions can be explained (or predicted) on the basis of the content of these states
%\label{ref:RNDLWEzQD7Ixq}(Dennett, 1991, p.76).
\parencite[][p.76]{dennett_consciousness_1991}. %
}
 The approach is clear, and amounts to correlating traces to directions and motivations in a~straightforward way. Of course, it is constrained to reading rational behavior and to valuing every action as instrumental to achieve a~specific goal. In contrast, a~neo-pragmatist view 
%\label{ref:RNDSIPGtsCWOd}(Brandom, 1994; 2000; Cash, 2008; 2009)
\parencites[][]{brandom_making_1994}[][]{brandom_articulating_2000}[][]{cash_thoughts_2008}[][]{cash_normativity_2009} %
 proceeds by ascribing intentionality as an explanation and a~specific coupling of action to social norms. As Cash 
%\label{ref:RNDyPGdQx45MP}(2008, p.101)
\parencite*[][p.101]{cash_thoughts_2008} %
 argues:
 \myquote{
 based on the similarity of their movement to the kind of actions, […] would entitle us to ascribe such intentional states as reasons.
 }
  This might be a~key aspect in certain contexts, but it is constrained to knowing what the norms to be applied are, and to evaluating if the ensuing pairing of actions to those norms succeed or not. In that sense, they imply the recognition of patterns, and a~judgment on their application or continuity, but they also underestimate the value of deviance and disregard a~space for individual creativity. Even a~third approach, which we can call a~neo-interactionist perspective, aims at understanding other's intentionalities not by acknowledging or judging their actions, but by understanding actual or potential interactions with others in socially appropriate ways. As Gallagher and Miyahara 
%\label{ref:RND9j2x1Jg8pS}(2012, p.135)
\parencite*[][p.135]{gallagher_neo-pragmatism_2012} %
 write in this account: 
 \myquote{
 we normally perceive another's intentionality in terms of its appropriateness, it's pragmatic and/or emotional value for our particular way of being, constituted by the particular goals or projects we have at the time, or implicit grasp on cultural norms, our social status, and so on, rather than as reflecting inner mental states, or as constituting explanatory reasons for her further thoughts and actions.
 }
 The neo-interactionist perspective certainly rounds up some of the forms in which intentionalities emerge, but they do not completely revoke the previous explanations, and instead helps compile a~catalogue of intentional enactions.

At this point, we can bring back the definition of an algorithm that was proposed in the first section, and mobilize it in an illustrative form. We can thus define an algorithm as

\myquote{
a~recursive script that links collective intentionalities with both material procedures and technocultural resources, enacted as a~cultural practice to accomplish a~specific task.

}
We now know what is meant by a~collective intentionality. And we can expect how to look for them. But this definition does more than just describe a~process. It wants to reflect on the fact that collective intentionalities are not by themselves the structures that sustain a~community's culture. It is really their mobilization, in an algorithmic form, which brings them to life. It would therefore be more precise to see an algorithm as an action than as an object, however ``unstable'' that object would turn out to be. On those grounds, an algorithm should be seen more as an activity, an ``algorithmation'', a~productive emergent pattern that enables connections out of a~given networked system or a~distinctive cognitive ecology.

Within the algorithmic feedback loops, the individual performs a~key computational function. Clearly, since we do not rely on a~machine-centered perspective, a~corresponding idea of computation must be outlined. For simplicity, we can take over Hutchins view here. He suggests that computation should be regarded as ``the propagation of representational states across representational media''
%\label{ref:RND21iPeY9KYo}(Hutchins, 1995, p.118).
\parencite[][p.118]{hutchins_cognition_1995}. %
 In that sense, individuals are the agents transforming representational states for those collective intentionalities through algorithmic procedures, that is, through recursive technical enactions. But in this finely threaded network, the individual is neither the origin nor the final end. And yet, she is not a~simple cog in the system either. She is interconnected, interacting, adjusting herself and her environment with this complex and finely tuned mechanism, which we might indeed call at this point a~socio-computing infrastructure 
%\label{ref:RND8YokZoYfJg}(Toscano, 2024).
\parencite[][]{toscano_intentionalities_nodate}. %
 Yet this term cannot imply a~fix and immovable architecture, but a~dynamic structure where certain accomplishments, and not others, are viable. Laland et al. 
%\label{ref:RNDesOUcdoMLM}(2000, p.130),
\parencite*[][p.130]{laland_niche_2000}, %
 for instance, refer as ``niche construction'' to the human-made or human customized structures that are essential to the development, production and continuance of certain activities. Jones et al. 
%\label{ref:RNDkiRMgz5mHh}(1997)
\parencite*[][]{clark_being_1997} %
 identify that same activity as ``ecosystem engineering''. The notion of socio-computing infrastructure that is proposed here here should be read along those lines, but where the accent on collective intentionalities and a~social cognitive activity is deliberate.

Similarly, Laland et al.
%\label{ref:RNDRFIIKPleNa}(2000)
\parencite*[][]{laland_niche_2000} %
 propose the idea of an ecological niche, which implies that an organism occupies a~distinctive role in each ecosystem. This opens up yet another approach to the task of identifying intentionalities, as a~supplement to the ones that were described above. For in certain contexts, defining the ecological niche of an individual can render a~better inspection of the collective intentionalities implied in a~given system. In other words, in human-machine interactional systems, focusing on the active or operating roles of given individuals as socially-enabled subjective behaviors or social functions can shed light on the specific collective intentionalities at work. This route acknowledges that the individual is relevant, only not on her own, but through her dynamic links (interpretations, associations, appreciations) to a~broader community of practice. This can be useful in anthropological cases, but is doubtlessly crucial in historical inquiries and techno-archeological analyses. We can bring a~couple of empirical cases from this latter for consideration.

\subsection*{a) Inka's Khipus}

If we think of historical socio-computing infrastructures that were lost or disrupted when the groups tied to them ceased to exist, we can acknowledge which were precisely the missing access points that make the reconstruction or re-interpretation of those systems difficult, or sometimes impossible. Two cases can be explored here at some length. As a~first case, we can recall the recent decipherment of ancient \textit{khipus} in Peru. Khipus were devices of statistical notation that stemmed during the Inka Empire, but were used until the Spanish colonial period in that South American country. These devices did not employ numerical symbols, but relied instead on cotton strings of different lengths and colors, and were encoded using knots at different places. As Medrano and Urton \parencite*[][p.2]{medrano_toward_2018} state: % describe:
\myquote{
the Inkas filled the twists and knots of the khipus with data, including bureaucratic accounting measures such as tax assignments and census counts.
%\label{ref:RNDmV4NfMXe9b}(Medrano and Urton, 2018, p.2).
%\parencite[][p.2]{medrano_toward_2018}. %
}
 One element is noticeable when approaching the khipu coded system: the people that used it as a~statistical artifact did not suddenly disappeared without leaving a~trace (as the Maya civilization did, for instance). On the contrary, the khipus coexisted during some time with the European statistical methods of the epoch, which the Spanish had brought with them. In that situation, even if symbolic and abstract operations were readily available, khipus were kept because they implied a~material manifestation of different social values, symbols that the people of that particular culture considered relevant information, as opposed to mere abstractions. In other words, khipus enriched merely numerical data: they registered social relations of a~highly organic and interpersonal nature, traits that were indifferent to the Spanish accounting methods, which were therefore inadequate for their transmission 
%\label{ref:RNDf0fim2T27j}(Medrano and Urton, 2018, p.12).
\parencite[][p.12]{medrano_toward_2018}.%


In the Inka worldview, khipus were not only statistics, but a~representation of a~given reality made possible through a~material craft. Of course, since the symbols they employed were not easily manageable, the khipus were discontinued after some time. Nobody wrote how they were encrypted, so the key to reading them disappeared. In a~sense, khipus were meant not only as notational systems, but also as mnemonic devices for khipu keepers and scribes. When these professionals finally changed the notational system to make their calculations, the mnemonic function ceased to operate. But while still active, these professionals were implementing an algorithmic procedure: they applied a~know-how for a~given collective intentionality---to count, or calculate, a~given state of affairs---and turned it into an objectified device---a social representation---thus computing it. The khipus were finally deciphered through an analogy with an European-style census that was later discovered to match one of these objects with a~strict correlation, but also by paying a~close attention to Inka's testimonies on economy, politics, religion and other aspects of their civilization that were highly valued, and considered to be worthy of a~specific notational foundation.

\subsection*{b) The Voynich Manuscript}

Another case is provided by the situation of the \textit{Voynich Manuscript}, kept in the collection of rare books at the Beinecke Library, at Yale. This fifteenth-century codex has not been deciphered until this day for several reasons, many of which are elements that indicate how a~socio-computational infrastructure, and with it a~specific algorithmic enaction, is put to work. The ``book'' was written in an unknown script by an unknown author. The impossibility to assign it a~context, a~precise culture, or even a~specific function within a~given literary or scientific biography, contribute to see this piece as an example of a~radical particularity that highlights its isolatory character. This is just not how a~``book'' works. Rather than executing a~typical communicative intentionality, the \textit{Voynich Manuscript} contradicts its form and function, and appears as a~work of madness. The current custodians of the book present it thus: ``the manuscript has no clearer purpose now than when it was rediscovered in 1912''
%\label{ref:RNDplYkHvycXg}(Clemens, 2016).
\parencite[][]{clemens_voynich_2016}. %
 There are no points of access because nobody knows where to begin with. Of course, some facts can be determined: the approximate date of its physical appearance, as well as a~list of its owners, all of which tempt the researchers to make some claims based on analogical and normative assumptions, of the kind that cognitive scientist have shown how to bring about. But in the end, the manuscript has been annulled as an informational device, as well as an instrument of contextual cognition. However, it has become a~new source of computations, for the curiosity of the researchers has turned it into an object of study, which means that it is being transfigured across different representational media. In any case, without an anchoring fact that stabilizes its meaning, such investigations speak more about our computational procedures than about the content of the ``book'', so they also tell about our need for conceptual pre-assumptions and our own inabilities to understand even human-made objects when a~clear intentionality is not recognized or set onto them.

\section*{Conclusions}
Algorithms cannot be reductively described as machine drivers or mere coding language. They imply instead a~complex cultural activity that involve both material and immaterial interactions. This article has aimed to show how, as part of their particular enaction, they are constructed along collective intentionalities of different sorts. In that sense, algorithms do shape desires, wants and needs, as these are ingrained in distinctive communities. It is indeed through an algorithmic recursiveness that collective cultures flourish and expand. It is also through an individual's tinkering with them that they can give way to adjustments and innovations, provided that the underlying intentionalities---whether as paths, patterns, occurrences or scripts---remain fundamentally recognizable.

In his book \textit{What do Algorithms want}?, Ed Finn finds an ingenuous answer to this complex question: ``This is what algorithms want, or what we design them to want: to know us completely''
%\label{ref:RNDDxaG3FGfbD}(Finn, 2017, p.82).
\parencite[][p.82]{finn_what_2017}. %
 But this statement is a~simplification that requires further clarification itself. Algorithms cannot want something in themselves, but neither do we. Or the other way around: algorithms want what ``we'' want, or rather: we want through them. Which is not always something evident. After all, ``to want'' is a~cultured habit, which is ingrained in children through upbringing and education. As individuals, we use socially available algorithms to channel pre-linguistic and abstract desires and needs, which only through them acquire a~definable form. So in a~way it is true: algorithms want what we design them to want. But we can only design what is culturally available, collectively interpretable, socially desirable. So it is less true that we design all algorithms ``to know us completely''. In fact, most of the time, the opposite is just the case. In their recursivity, algorithms enact collective intentionalities that are frequently turned into latencies, background or naturalized conventions, and then cease to appear as constructions to us. (Therefore, only in a~culture where information extraction is a~viable practice, the design of algorithms to extract information from us---what Finn refers as ``to know''---will be a~logical consequence.) In the end, algorithms imply an articulatory activity: they are collective processes of cultural inscription, through which individuals enact socially available programmatic technologies for a~specific, intentional objective.

This article has sought to provide examples on how to approach collective intentionalities, both by recalling how cognitive scientists apply logical inferences to distinguish emergent phenomena, and by turning to historical socio-computing infrastructures to inspect their legibility (or lack thereof) and operation. Evidently, much works needs to be done to deepen a~techno-archeological inquiry of this kind, but this article has sought to contribute with some entry points to enrich such analyses in a~distinctive way.



\end{artengenv}

